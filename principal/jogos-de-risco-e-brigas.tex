{\let\clearpage\relax \chapter{Jogos de Risco e Brigas Generalizadas}}

\begin{article}
	Não haverá punição para entidades no caso de briga isolada, seja entre torcedores, seja entre atletas.
\end{article}

\begin{article}
	Um jogo será considerado de risco em reunião, ordinária ou extraordinária, antes do jogo. O jogo é de risco quando uma das Faculdades envolvidas solicitar e pelo menos metade da Comissão Organizadora aprovar ou quando a maioria simples da Comissão Organizadora assim achar conveniente, sendo que as envolvidas votam nas duas ocasiões.

	\begin{xparagraph}
		Em caso de empate na votação da Comissão Organizadora o jogo será considerado de risco.
	\end{xparagraph}

	\begin{xparagraph}
		As entidades envolvidas no jogo deverão designar 5 (cinco) representantes para compor a Comissão Organizadora no jogo de risco.
	\end{xparagraph}

	\begin{xparagraph}
		As entidades não envolvidas no jogo deverão designar 2 (dois) representantes para compor a Comissão Organizadora no jogo
	\end{xparagraph}

	\begin{xparagraph}
		Caso seja requerido por qualquer entidade e aprovado em votação pela maioria simples da Comissão Organizadora, poder-se-á diminuir o número de representantes ao qual se referem os parágrafos anteriores para 3 (três) representantes por entidade envolvida no jogo e 1 (um) representante por entidade não envolvida no jogo.
	\end{xparagraph}

	\begin{xparagraph}
		Todos os representantes das entidades designados para o jogo deverão se identificar para o representante oficial do jogo em questão, que deverá organizar uma lista de presença, que deverá ser assinada no início do jogo e no término do mesmo. Caso o representante do jogo de risco não compareça, cabe a própria Comissão Organizadora providenciar a lista.
	\end{xparagraph}

	\begin{xparagraph}
		Caso alguma entidade não compareça no local e horário determinado para a partida com o número de representantes designados, será considerado W.O. de representação, sendo aplicada uma multa pecuniária de 1/4 (um quarto) do salário mínimo para cada representante ausente.
	\end{xparagraph}
\end{article}

\begin{article}
	Em caso de briga generalizada entre torcidas, as entidades envolvidas podem ser punidas da seguinte forma:

	\begin{enumerate}[noitemsep]
		\item Multa pecuniária de 2 (dois) salários mínimos e impedimento de levar torcida ao próximo jogo;
		\item Multa pecuniária de 4 (quatro) salários mínimos e impedimento de levar torcida ao próximo jogo, se o jogo foi previamente considerado de risco.
	\end{enumerate}

	\begin{xparagraph}
		O impedimento de torcida deverá ser aplicado no 1o jogo dentre todas as modalidades de quadra ou Futebol de Campo, subsequente ao dia em que a briga ocorreu. As punições de impedimento de torcida deverão ser aplicadas mesmo que o próximo jogo ocorra no ano seguinte.
	\end{xparagraph}
\end{article}

\begin{article}
	A quadra se tornará área da torcida da entidade vencedora do jogo somente nas partidas não consideradas de risco, desde que seja o último jogo da praça esportiva no dia.

	\begin{xparagraph}
		A entidade vencedora deverá deixar a quadra em plenas condições de jogo num prazo máximo de 30 (trinta) minutos imediatamente após o término da partida. O representante oficial deve, sempre que solicitado, informar aos membros da entidade responsável.
	\end{xparagraph}

	\begin{xparagraph}
		Caso a entidade em questão não deixe a quadra em condições de jogo no prazo máximo, será punida com multa pecuniária de 1⁄2 (meio) salário mínimo.
	\end{xparagraph}
\end{article}

\begin{article}
	Caso a torcida de uma atlética invada a quadra após um jogo de risco, a entidade representante da torcida será punida com impossibilidade de levar torcida no próximo jogo de risco e multa pecuniária de 2 (dois) salários mínimos.

	\begin{xparagraph}
		As disposições da "caput" deste artigo não se aplicam caso a Atlética seja campeã geral e o jogo seja o último da praça esportiva do dia.
	\end{xparagraph}

	\begin{xparagraph}
		A invasão de quadra deverá ser feita 5 (cinco) minutos após o término do jogo para que haja evacuação dos atletas da quadra e deverá ser feito cordão de isolamento pela Comissão Organizadora
	\end{xparagraph}
\end{article}

\begin{article}
	Em jogos que não forem considerados de risco, deve-se utilizar o relatório do árbitro e do representante para que seja considerada uma briga como generalizada. É com base nesses relatórios que a Comissão Organizadora julgará se alguma das entidade envolvidas será ou não punida com multa pecuniária de 2 (dois) salários mínimos e impedimento de levar torcida no próximo jogo.
\end{article}
