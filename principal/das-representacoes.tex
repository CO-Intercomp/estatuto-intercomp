{\let\clearpage\relax \chapter{Das Representações}}

\begin{article}
	Caberá às Atléticas se responsabilizarem pelas representações das partidas e confrontos de todas as modalidades da INTERCOMP.
\end{article}

\begin{article}
	A CO Executiva deverá fazer uma conferência técnica com a equipe de arbitragem ao início da disputa de cada modalidade da INTERCOMP para diluir as dúvidas sobre aspectos gerais da competição, seja no que diz respeito a horários até a forma de disputa da modalidade em questão, sempre baseado no regulamento em vigor.

	\begin{xparagraph}
		Devem estar no banco de reservas somente os atletas reservas e a comissão técnica.
	\end{xparagraph}

	\begin{xparagraph}
		A comissão técnica poderá ser composta, no máximo, pelo treinador da equipe, auxiliar técnico e representante da entidade em questão que deverão estar inscritos na súmula do jogo, mediante apresentação de documentos oficiais.
	\end{xparagraph}

	\begin{xparagraph}
		O mesário será responsável pela verificação dos documentos. Caso o documento esteja em desacordo com o regulamento, caberá ao mesário vetar a participação do atleta na disputa.
	\end{xparagraph}

	\begin{xparagraph}
		Todas as súmulas serão recolhidas pelo representante esportivo das entidades, para conferência em reunião da CO ao final do dia.
	\end{xparagraph}
\end{article}

\begin{article}
	É vetado ao representante arbitrar jogos.
\end{article}
