
{\let\clearpage\relax \chapter{Das Representações e dos Documentos Oficiais}}

\begin{article}
	Para a oficialização da participação da entidade nos jogos e competições da \textbf{INTERCOMP} é necessário a entrega da listagem oficial de alunos atletas contendo o nome completo, o número de Registro de Aluno ou de Registro Acadêmico e o curso superior em que está matriculado.
\end{article}

\begin{article}
	Os representantes de cada partida devem conferir as credenciais e documentos dos times (respeitando o limite de tolerância de atraso e as restrições) e assinar a folha de representação até o horário do jogo que está na tabela, mesmo que este esteja atrasado. Será caracterizado W.O. de representação caso o representante não assine a folha de representação até o horário de tabela do jogo.

	\begin{xparagraph}
		A folha de representação também deverá ser assinada logo ao término do jogo, sendo caracterizado W.O. de representação caso isso não ocorra.
	\end{xparagraph}

	\begin{xparagraph}
		As Entidades não precisam mandar representantes para as modalidades em que elas não participam.
	\end{xparagraph}
\end{article}

\begin{article}
	A representação deverá ser feita por uma entidade. Nas finais, a representação deverá ser feita por todas as Entidades, sendo definida uma Entidade, por sorteio em reunião, responsável pelo preenchimento da folha de representação.

	\begin{xparagraph}
		Todos os representantes devem assinar as folhas de representação, constando o horário em que assinaram. Definir as representações das finais dos jogos na reunião noturna que ocorre de sábado para domingo no final de semana da INTERCOMP.
.
	\end{xparagraph}

	\begin{xparagraph}
		Este artigo não se aplica a modalidades individuais, a exceção do tênis, pois nelas apenas os representantes das delegações que alcançaram as semi-finais precisam comparecer às finais. As demais entidades estão dispensadas assim que forem eliminadas da modalidade.
	\end{xparagraph}
\end{article}

\begin{article}
	Caberá à Comissão Organizadora da competição designar representantes das entidades para todas as modalidades e confrontos da \textbf{INTERCOMP}.
\end{article}

\begin{article}
	Os representantes devem estar devidamente identificados no formato deliberado pela Comissão Organizadora.
\end{article}

\begin{article}
    \label{art:representantedocumentos}
	O representante deve examinar os documentos dos atletas inscritos nas modalidades de acordo com o estabelecido neste Estatuto.

	\begin{xparagraph}
	    Poderá o representante solicitar, a qualquer momento, a identificação de qualquer atleta, sendo está feita através de documentos especificados no art. \ref{art:documentosoficiais}\ulo.
	\end{xparagraph}

	\begin{xparagraph}
	    Caso o atleta não apresente um documento oficial, cabe ao representante impedi-lo de participar do jogo.
	\end{xparagraph}
\end{article}

\begin{article}
	Nas modalidades onde existe arbitragem oficial, caberá ao representante indicado para a partida esclarecer as dúvidas sobre aspectos gerais da competição, seja no que diz respeito a horários até a forma de disputa da modalidade em questão, sempre baseado no estatuto em vigor.
\end{article}

\begin{article}
	O representante deverá certificar-se de que no banco de reservas estejam somente os atletas reservas, a comissão técnica e os profissionais da saúde.

	\begin{xparagraph}
	    A comissão técnica poderá ser composta, pelo treinador da equipe e auxiliar técnico. Estes deverão estar inscritos na súmula do árbitro, mediante apresentação de documentos oficiais. Se faz a necessidade da apresentação de documento oficial dos profissionais da saúde presentes no banco.
	\end{xparagraph}
\end{article}

\begin{article}
	Se o representante possuir dúvidas no decorrer da partida, este deverá deixá-la prosseguir sem interrupção, fazer um relatório por escrito e entregá-lo na próxima reunião da Comissão Organizadora, onde será tomada a devida providência.
\end{article}

\begin{article}
	É vedado ao representante arbitrar jogos.
\end{article}

\begin{article}
	Nenhuma Entidade pode representar os seus próprios jogos, salvo caso onde todas as Entidades sejam representantes do jogo. Ressalta-se que, nas finais das modalidades coletivas, as Entidades envolvidas na partida não poderão ser representantes oficiais, devendo ser suplentes.
\end{article}

\begin{article}
	Apenas membros da comissão técnica e capitão têm direito a relatar algo na folha de representação, além dos próprios representantes.
\end{article}

\begin{article}
    \label{art:documentosoficiais}
	Considera-se documento oficial: Carteira de Identidade (nacional), Documento Nacional de Identificação, passaporte (nacional ou internacional), Carteira Nacional de Habilitação com foto, Carteira de Trabalho, Registro Nacional de Estrangeiro, R.E. Militar e Boletim de Ocorrência.

	\begin{xparagraph}
	    Além dos documentos originais, também são considerados documentos oficiais cópias autenticadas em cartório dos documentos citados acima.
	\end{xparagraph}

	\begin{xparagraph}
	    O Boletim de Ocorrência só será aceito como documento oficial se o atleta já tiver sido inscrito em súmula em um jogo de qualquer modalidade da edição, devendo ser apresentado em conjunto com um documento com foto não removível.
	\end{xparagraph}

	\begin{xparagraph}
	    O documento que estiver com data de validade expirada será considerado válido.
	\end{xparagraph}
\end{article}