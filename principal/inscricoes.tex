
{\let\clearpage\relax \chapter{Inscrições}}

\begin{article}
	Poderão participar da \textbf{INTERCOMP} somente os alunos dos cursos de graduação e pós-graduação Strictus das Faculdades participantes relacionadas no art. \ref{art:integrantes}\ulo\ deste estatuto, que tiverem o seu nome na lista oficial de alunos expedida pela secretaria da respectiva escola.

	\begin{xparagraph}
		É vetada a participação de alunos especiais ou com matrícula trancada desde o início do ano letivo.
	\end{xparagraph}

	\begin{xparagraph}
		O aluno que estiver matriculado em dois ou mais Institutos de Ensino Superior só poderá jogar por um deles durante toda a competição.
	\end{xparagraph}

	\begin{xparagraph}
		Alunos intercambistas, mediante comprovação de documentos e comprovação de participar de cursos estatutários apenas.
	\end{xparagraph}
\end{article}

\begin{article}
	Cada entidade deverá entregar, em data marcada pela Comissão Organizadora, uma lista de todos os alunos de sua Faculdade.

	\begin{xparagraph}
		Sendo computadorizada (formulário contínuo ou páginas numeradas), deverá ser assinada pelo diretor da faculdade ou outro responsável aprovado pela Comissão Organizadora, na primeira e na última folha, com firma reconhecida em ambas as assinaturas.
	\end{xparagraph}

	\begin{xparagraph}
		Não sendo computadorizada, a listagem deve ser apresentada em papel timbrado da faculdade, devendo ser assinada pelo diretor da faculdade ou outro responsável aprovado pela Comissão Organizadora, em todas as folhas, com firma reconhecida em todas as assinaturas.
	\end{xparagraph}

	\begin{xparagraph}
		Sendo computadorizada e com páginas não numeradas, deverá ter todas as folhas assinadas pelo diretor ou outro responsável aprovado pela Comissão Organizadora, com firma reconhecida na primeira e na última folha.
	\end{xparagraph}

	\begin{xparagraph}
		As entidades participantes devem enviar para a comissão organizadora uma listagem devidamente homologada pela instituição de ensino, contendo a discriminação dos alunos regularmente matriculados na instituição.
	\end{xparagraph}

	\begin{xparagraph}
		A entidade pagará uma multa de \sfrac{1}{2} salário mínimo (usando como base o salário mínimo vigente no momento do pagamento) por atraso (baseado na data estipulada pela Comissão Organizadora) na entrega de cada um dos documentos que forem necessários na inscrição, exceto por casos extraordinários que deverão ser julgados pela CO.
	\end{xparagraph}

	\begin{xparagraph}
		As listagens devem ser aprovadas pela Comissão Organizadora no ato da apresentação da listagem. No caso da listagem não ser aprovada pela Comissão Organizadora, a entidade deverá entregar nova listagem em nova data estipulada pela CO e deverá pagar a multa estipulada no parágrafo anterior até a realização da próxima competição.
	\end{xparagraph}
\end{article}

\begin{article}
	Para participar da \textbf{INTERCOMP} é obrigatório a entrega do Termo de Compromisso Geral fornecido pela CO.

	\begin{xparagraph}
		Cada entidade se responsabiliza por entregar o termo de compromisso geral da \textbf{INTERCOMP} à todos os atletas da sua delegação e entregar os respectivos termos assinados para a CO.
	\end{xparagraph}
\end{article}

\begin{article}
	Cada atleta deverá ser inscrito na súmula momentos antes do início do jogo, mediante comprovação de seu vínculo com a IES participante.

	\begin{xparagraph}
		Cabe à Comissão Organizadora vigente definir os meios para avaliar se cada atleta está apto, ou não, a participar dos jogos.
	\end{xparagraph}
\end{article}

\begin{article}
	Todo atleta deverá apresentar em cada jogo um documento oficial (original) com foto.

	\begin{xparagraph}
		A ausência do documento impossibilita o atleta de participar da competição.
	\end{xparagraph}

	\begin{xparagraph}
		Considera-se documento oficial a carteira de identidade, passaporte, R.E. Militar, RNE e Carteira Nacional de Habilitação com foto.
	\end{xparagraph}

	\begin{xparagraph}
		Na ausência dos documentos listados acima, havendo comum acordo entre as duas equipes, e apresentando carteirinha da faculdade original com foto e nome do jogador, o jogo poderá ocorrer e será avaliado pela CO.
	\end{xparagraph}
\end{article}

\begin{article}
	Será permitido que uma entidade inscreva atletas formados há até 2 (dois) anos e/ou que sejam alunos de um de seus cursos não estatutários, conforme o art. \ref{art:integrantes}\ulo. A cada confronto de cada modalidade, este número não poderá superar o número de atletas inscritos naquele confronto que atualmente cursam um de seus cursos estatutários.

	\begin{xparagraph}
		Além do disposto no caput deste artigo, o número de formandos inscritos em cada partida não poderá superar 2 (dois)
	\end{xparagraph}

	\begin{xparagraph}
		Alunos formados deverão apresentar certificado de conclusão de curso junto a seu documento pessoal para inscrição nas partidas.
	\end{xparagraph}

	\begin{xparagraph}
		Caso seja verificado que uma entidade inscreveu um número de alunos de cursos estatutários inferior ao de formandos e cursos não estatutários, o resultado da partida em questão será anulado e considerar-se-á W.O. dessa entidade.
	\end{xparagraph}
\end{article}
