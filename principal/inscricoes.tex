
{\let\clearpage\relax \chapter{Inscrições}}

\begin{article}
	Poderão participar da \textbf{INTERCOMP} somente os alunos dos cursos de graduação e pós-graduação Strictus das Faculdades participantes relacionadas no art. \ref{art:integrantes}\ulo\ deste estatuto, que tiverem o seu nome na lista oficial de alunos expedida pela secretaria da respectiva escola.

	\begin{xparagraph}
		É vetada a participação de alunos especiais ou com matrícula trancada desde o início do ano letivo.
	\end{xparagraph}

	\begin{xparagraph}
		O aluno que estiver matriculado em dois ou mais Institutos de Ensino Superior só poderá jogar por um deles durante toda a competição.
	\end{xparagraph}

	\begin{xparagraph}
		Alunos intercambistas, mediante comprovação de documentos e comprovação de participar de cursos estatutários apenas.
	\end{xparagraph}
\end{article}

\begin{article}
	Caso seja verificado que uma entidade inscreveu um número de alunos de cursos estatutários inferior ao de formandos e cursos não estatutários, o resultado da partida em questão será anulado e considerar-se-á W.O. dessa entidade.

\end{article}

\begin{article}
	Para participar da \textbf{INTERCOMP} é obrigatório a entrega do Termo de Compromisso Geral fornecido pela Comissão Organizadora.

	\begin{xparagraph}
		Cada entidade se responsabiliza por entregar o Termo de Compromisso Geral da \textbf{INTERCOMP} à todos os atletas da sua delegação e entregar os respectivos termos assinados para a Comissão Organizadora.
	\end{xparagraph}
\end{article}

\begin{article}
    \label{art:listagemoficial}
	Para a oficialização da participação da entidade nos jogos e competições da \textbf{INTERCOMP} é necessário a entrega da listagem oficial de alunos atletas contendo o nome completo, o número de Registro de Aluno ou de Registro Acadêmico e o curso superior em que está matriculado. Além dessa é opcional a entrega de uma segunda lista com uma alteração de no máximo metade dos atletas. Essa segunda entrega somente deve ser realizada mediante a alteração da lista e deve ser entregue até 7 dias antes do evento.

	\begin{xparagraph}
		Sendo computadorizada (formulário contínuo ou páginas numeradas), deverá ser reconhecido pelo órgão responsável da faculdade, com assinatura ou carimbo na primeira e na última folha.
	\end{xparagraph}

	\begin{xparagraph}
		Não sendo computadorizada, a listagem deve ser apresentada em papel timbrado da Faculdade, devendo ser reconhecido pelo órgão responsável da faculdade, com assinatura ou carimbo em todas as folhas.
	\end{xparagraph}

	\begin{xparagraph}
		Sendo computadorizada e com páginas não numeradas, devendo ser reconhecido pelo órgão responsável da faculdade, com assinatura ou carimbo em todas as folhas.
	\end{xparagraph}

	\begin{xparagraph}
    	As entidades participantes devem enviar para a comissão organizadora uma listagem devidamente homologada pela instituição de ensino, contendo a discriminação dos alunos regularmente matriculados na instituição.
	\end{xparagraph}

	\begin{xparagraph}
	    A entidade pagará uma multa de \sfrac{1}{2} salário mínimo (usando como base o salário mínimo vigente no momento do pagamento) por atraso (baseado na data estipulada pela Comissão Organizadora) na entrega de cada um dos documentos que forem necessários na inscrição, exceto por casos extraordinários que deverão ser julgados pela Comissão Organizadora.
	\end{xparagraph}

	\begin{xparagraph}
	    As listagens devem ser aprovadas pela Comissão Organizadora no ato da apresentação da listagem. No caso da listagem não ser aprovada pela Comissão Organizadora, a entidade deverá entregar nova listagem em nova data estipulada pela Comissão Organizadora e deverá pagar a multa estipulada no parágrafo anterior até a realização da próxima competição.
	\end{xparagraph}

	\begin{xparagraph}
	    Além da listagem cada Entidade deverá preencher uma planilha de inscrição de atletas fornecida pela Presidência da Comissão Organizadora.
	\end{xparagraph}

	\begin{xparagraph}
	    A entrega dos documentos descritos neste artigo deve ser feita até 10 dias antes do início do evento.
	\end{xparagraph}
\end{article}

\begin{article}
    \label{art:credencialatleta}
    Denomina-se Credencial de Atleta o documento emitido pela Comissão Organizadora para cada atleta inscrito e que contenha, pelo menos, o nome completo e a instituição de ensino superior do atleta, para que sejam utilizadas para a identificação de atletas durante o evento.

	\begin{xparagraph}
	    As credenciais de atletas de cursos não-estatutários e formados devem ser diferenciadas das credenciais de atletas de cursos estatutários.
	\end{xparagraph}

	\begin{xparagraph}
	    O modelo da Credencial de Atleta deverá ser deliberado pela Comissão Organizadora até 10 dias antes do evento.
	\end{xparagraph}
\end{article}

\begin{article}
    Denomina-se o Documento de Inscrição de Atletas o documento que reúne as seguintes informações de todas as Entidades:
	\begin{enumerate}[noitemsep,leftmargin=2\parindent]
		\item Documentos descritos no art. \ref{art:listagemoficial}\ulo
		\item Credenciais de Atletas, descritas no art. \ref{art:credencialatleta}\ulo
	\end{enumerate}
\end{article}

\begin{article}
    \label{art:inscricaoatletas}
    A Presidência da Comissão Organizadora deverá gerar o Documento de Inscrição de Atletas e enviar para os presidentes de cada uma das Entidades.

	\begin{xparagraph}
	    O envio do Documento de Inscrição de Atletas pode ser feito tanto no meio físico quanto no meio digital, conforme for deliberado pela Comissão Organizadora.
	\end{xparagraph}

	\begin{xparagraph}
	    O envio do Documento de Inscrição de Atletas descrito neste artigo deve ser feito até 14 dias antes do evento. Para atletas discriminados na segunda lista, o envio deve ser feito até 3 dias antes do evento.
	\end{xparagraph}
\end{article}

\begin{article}
    \label{art:presidentesassinar}
    Uma vez enviado, é de responsabilidade do Presidente de cada Entidade conferir as informações do Documento de Inscrição de Atletas e assiná-lo para confirmar a validade do documento ou apontar quaisquer erros para que sejam corrigidos pela Comissão Organizadora.

	\begin{xparagraph}
	    A assinatura do Documento de Inscrição de Atletas deve ser feito no mesmo meio (físico ou digital) que o seu envio, de acordo com o art. \ref{art:inscricaoatletas}\ulo.
	\end{xparagraph}

	\begin{xparagraph}
        Cada Entidade será responsável por validar as credenciais e os documentos de, pelo menos, 1 (uma) outra Entidade, a ser definido por sorteio pela Comissão Organizadora.
	\end{xparagraph}

	\begin{xparagraph}
        A Entidade que não validar ou que aprovar os documentos e credenciais de atletas irregulares da outra Entidade para qual for sorteada poderá ser multada pela Comissão Organizadora em 1 (um) pacote base por atleta irregular inscrito.
	\end{xparagraph}

	\begin{xparagraph}
	    A assinatura do Documento de Inscrição de Atletas por parte das Entidades deve ser feito até 7 dias antes do evento.
	\end{xparagraph}
\end{article}

\begin{article}
    Uma vez que o Documento de Inscrição de Atletas tenha sido devidamente conferido e assinado, de acordo com o art. \ref{art:presidentesassinar}\ulo, a Comissão Organizadora deverá enviar as Credenciais de Atleta geradas para que cada Entidade possa distribuí-las entre seus atletas inscritos.

	\begin{xparagraph}
	    O envio das Credenciais de Atleta pode ser feito tanto no meio físico quanto no meio digital, conforme for deliberado pela Comissão Organizadora.
	\end{xparagraph}
\end{article}

\begin{article}
    Cada atleta deverá ser inscrito na súmula momentos antes do início do jogo, mediante entrega de:
	\begin{enumerate}[noitemsep,leftmargin=2\parindent]
		\item Sua Credencial de Atleta emitida pela Comissão Organizadora
		\item Um documento oficial com foto, conforme descrito no art. \ref{art:documentosoficiais}\ulo
	\end{enumerate}

	\begin{xparagraph}
	     A não entrega de um dos dois documentos poderá impedir o atleta de disputar a partida, conforme descrito no art. \ref{art:representantedocumentos}\ulo.
	\end{xparagraph}
\end{article}

\begin{article}
    Caso um atleta não apresente a credencial, ou apresente a credencial incorreta, ele poderá jogar a partida, desde que o representante oficial da partida descreva detalhadamente o erro da credencial ou sua ausência em súmula de representação, e que a Entidade do atleta em questão leve a credencial correta na primeira reunião da Comissão Organizadora que for realizada após o ocorrido, com exceção do último dia, em que não serão aceitos atletas sem credenciais ou com erros nesta.

	\begin{xparagraph}
	    O número máximo, por súmula, de atletas aceitos sem credencial ou com credencial incorreta é dois, não podendo exceder essa quantidade.
	\end{xparagraph}

	\begin{xparagraph}
	    Após apresentação e conferência pela Comissão Organizadora da(s) credencial(is) correta(s) em reunião, o atleta é definido como regular no jogo em questão.
	\end{xparagraph}

	\begin{xparagraph}
	    Caso após o recredenciamento, a credencial continue errada, a Entidade em questão será desclassificada por WO e será penalizada com o valor de um pacote base da edição vigente.
	\end{xparagraph}
\end{article}

\begin{article}
    Todas as Entidades deverão levar na reunião a listagem geral de alunos da IES que está credenciando.
\end{article}

\begin{article}
    Caso uma Entidade for suspeita de ter jogadores irregulares em algum de seus times, essa deverá apresentar toda a documentação do atleta que a Comissão Organizadora solicitar.
\end{article}
