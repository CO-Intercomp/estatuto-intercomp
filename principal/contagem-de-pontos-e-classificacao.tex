{\let\clearpage\relax \chapter{Contagem de Pontos e Classificação}}

\begin{article}
	A contagem de pontos da \textbf{INTERCOMP} será feita por modalidade e atribuindo-se pontos, de acordo com a sua classificação, a pontuação deverá ser:
	\begin{description}[noitemsep]
		\item[1\ulo\ colocado] 20 pontos
		\item[2\ulo\ colocado] 16 pontos
		\item[3\ulo\ colocado] 13 pontos
		\item[4\ulo\ colocado] 11 pontos
		\item[5\ulo\ colocado] 09 pontos
		\item[6\ulo\ colocado] 07 pontos
		\item[7\ulo\ colocado] 06 pontos
		\item[8\ulo\ colocado] 04 pontos
		\item[9\ulo\ colocado] 03 pontos
		\item[10\ulo\ colocado] 02 pontos
		\item[11\ulo\ colocado] 01 pontos
		\item[12\ulo\ colocado] 01 pontos
	\end{description}

	\begin{xparagraph}
		Para efeito de pontuação considerar-se-á que a primeira colocada da segunda divisão fica classificada imediatamente após a última colocada da primeira divisão.
	\end{xparagraph}

	\begin{xparagraph}
		As categorias masculina e feminina da modalidade Truco não contarão ponto para a classificação geral, e sim para uma classificação geral do Truco, seguindo a mesma distribuição de pontos. Na classificação geral será contabilizado apenas classificação geral do Truco.
	\end{xparagraph}

	\begin{xparagraph}
		No caso de modalidades nas quais competiram 12 equipes ou menos, todas as equipes pontuarão de acordo com sua colocação. No caso de competições com mais de 12 equipes, somente as 12 primeiras pontuarão.
	\end{xparagraph}
\end{article}

\begin{article}
	Será declarada a vencedora do \textbf{INTERCOMP} a entidade que totalizar o maior número de pontos.

	\begin{xparagraph}
		Em caso de empate na contagem geral, será declarada vencedora a entidade possuidora de maior número de primeiros lugares. Se persistir o empate, recorrer-se-á às seguintes colocações e assim por diante.
	\end{xparagraph}

	\begin{xparagraph}
		Persistindo o empate a vencedora do maior número de confrontos diretos entre as empatadas, com exceção de modalidades disputadas como \textit{Torneio por Provas}, será considerada campeã.
	\end{xparagraph}

	\begin{xparagraph}
		Na hipótese de se manter o empate, ambas as entidades serão proclamadas campeãs.
	\end{xparagraph}
\end{article}
