{\let\clearpage\relax \chapter{Forma de Disputa}}

\noindent
Ficam definidos como cabeças de chave de cada modalidade as 4 (quatro) equipes melhor classificadas naquela modalidade na edição última do Intercomp
\begin{itemize}[noitemsep]
	\item Caso a modalidade não tenha sido disputada na última edição do Intercomp, considerar-se-ão as classificações gerais da última edição do Intercomp
\end{itemize}

\noindent
Fica definido como \textit{Disputa em Eliminatórias Simples} o seguinte sistema de diputa::
\begin{itemize}[noitemsep]
	\item Os participantes são distribuídos em posições ordenadas de 1 a 16:
	\begin{itemize}[noitemsep]
		\item As equipes cabeças de chave são alocadas da seguinte forma:
		\begin{itemize}[noitemsep]
			\item A primeira cabeça de chave ocupa a posição 1
			\item A segunda cabeça de chave ocupa a posição 16
			\item A terceira cabeça de chave ocupa a posição 9
			\item A quarta cabeça de chave ocupa a posição 8
		\end{itemize}

		\item As demais equipes tem sua posição definida por sorteio, da seguinte forma:
		\begin{itemize}[noitemsep]
			\item A primeira sorteada ocupa a posição 5
			\item A segunda sorteada ocupa a posição 12
			\item A terceira sorteada ocupa a posição 13
			\item A quarta sorteada ocupa a posição 4
			\item A quinta sorteada ocupa a posição 6
			\item A sexta sorteada ocupa a posição 11
			\item A sétima sorteada ocupa a posição 14
			\item A oitava sorteada ocupa a posição 3
			\item A nona sorteada ocupa a posição 7
			\item A décima sorteada ocupa a posição 10
			\item A décima primeira sorteada ocupa a posição 15
			\item A décima segunda sorteado ocupa a posição 2
		\end{itemize}

		\item Sendo o número n de participantes menor que 16, os 16 - n últimos sorteios não são realizados, as posições que seriam preenchidas por estes sorteios ficam vagas e são denominadas \textit{BYE}.
	\end{itemize}

	\item As participantes se enfrentam em confrontos eliminatórios em quatro fases: oitavas-de-final, quartas-de-final, semifinais e final. As perdedoras de cada confronto são eliminadas.
	\begin{itemize}[noitemsep]
		\item O confronto de oitavas-de-final O1 se dá entre os participantes que ocuparem as posições 1 e 2, o confronto O2 entre os participantes que ocuparem as posições 3 e 4, e assim por diante.
		\item O confronto de quartas-de-final Q1 se dá entre os vencedores dos confrontos O1 e O2, o confronto Q2 se dá entre os vencedores dos confrontos O3 e O4, e assim por diante.
		\item O confronto semifinal S1 se dá entre os vencedores dos confrontos Q1 e Q2, o confronto S2 entre os vencedores dos confrontos Q3 e Q4.
		\item O confronto final F1 se dá entre os vencedores dos confrontos S1 e S2
	\end{itemize}

	\item Caso o confronto de uma equipe seja contra um BYE, a equipe é automaticamente declarada vencedora do confronto.
	\item Caso ocorra confronto entre dois BYEs, o vencedor do confronto é denominado BYE
\end{itemize}

\noindent
Fica definido como \textit{Disputa em Grupos} a seguinte forma de disputa:
\begin{itemize}[noitemsep]
	\item As participantes são distribuídos em grupos de acordo com a quantidade de participantes inscritos:
	\begin{itemize}[noitemsep]
		\item Havendo até 5 participantes inscritas, é formado um grupo único contendo todas as paticipantes

		\item Havendo entre 6 e 8 participantes inscritos, são formados dois grupos:
		\begin{itemize}[noitemsep]
			\item O grupo A com metade da quantidade de participantes arredondada para baixo, contendo a primeira e a quarta cabeças de chave;
			\item O grupo B com a quantidade de participantes restante, contendo a segunda e a terceira cabeças de chave
		\end{itemize}

		\item Havendo entre 9 e 12 participantes inscritas, são formados três grupos:
		\begin{itemize}[noitemsep]
			\item O grupo A com um terço da quantidade de participantes arredondada para baixo, contendo a primeiro cabeça de chave;
			\item O grupo B com metade da quantidade de participantes restante arredondada para baixo, contendo a segundo cabeça de chave;
			\item O grupo C com a quantidade de participantes restante, contendo a terceira e a quarta cabeças de chave.
		\end{itemize}

		\item Havendo entre 13 e 16 participantes inscritas, são formados quatro grupos:
		\begin{itemize}[noitemsep]
			\item O grupo A com um quarto da quantidade de participantes arredondada para baixo, contendo o primeira cabeça de chave;
			\item O grupo B com um terço da quantidade de participantes arredondada para baixo, contendo a segunda cabeça de chave;
			\item O grupo C com metade da quantidade de participantes restante arredondada para baixo, contendo a terceira cabeça de chave;
			\item O grupo D com a quantidade de participantes restante, contendo a quarta cabeça de chave.
		\end{itemize}
	\end{itemize}

	\item Cada participante de cada grupo joga um confronto contra cada um dos outros participantes do seu grupo
	\item As participantes são ordenados da seguinte forma:
	\begin{itemize}[noitemsep]
		\item Para cada grupo, é considerada \textit{Campeã do Grupo} a participante com melhor desempenho de acordo com os \textit{Critérios de Classificação} da modalidade.
		\item As \textit{Campeãs dos Grupos} são ordenadas a frente das demais equipes, e ordenadas entre sí seguindo os \textit{Critérios de Classificação} da modalidade, de acordo com os resultados que obtiveram em seus respectivos grupos.
		\item As demais participantes são ordenadas entre sí seguindo os \textit{Critérios de Classificação} da modalidade, de acordo com os resultados que obtiveram em seus respectivos grupos.
	\end{itemize}
\end{itemize}

\noindent
\noindent
Será realizado sorteio para alocação das participantes que não são cabeças de chave entre os grupos para as modalidades que envolvem \textit{Disputas em Grupos}, da seguinte forma:
\begin{itemize}[noitemsep]
	\item Havendo grupo único, não será realizado sorteio
	\item Havendo 2 (dois) grupos, as participantes a serem sorteadas são colocadas em um \textit{Pote Único}, do qual serão sortedas para ocupar as posições restantes em cada grupo
	\item Havendo 3 (três) grupos, as duas participantes a serem sorteadas mais bem classificadas de acordo com as regras para definição de cabeças de chave do artigo X serão colocadas no \textit{Pote 1}, do qual serão sorteadas uma para o grupo A e a outra para o grupo B. As demais participantes a serem sorteadas serão colocadas no \textit{Pote 2}, do qual serão sorteadas para ocupar as posições restantes em cada grupo.
	\item Havendo 4 (quatro) grupos, as quatro participantes a serem sorteadas mais bem classificadas de acordo com as regras para definição de cabeças de chave do artigo X serão colocadas no \textit{Pote 1}, do qual serão sorteadas cada uma para um grupo. As demais participantes a serem sorteadas serão colocadas no \textit{Pote 2}, do qual serão sorteadas para ocupar as posições restantes em cada grupo.
\end{itemize}

\noindent
Fica definido como \textit{Torneio de Eliminação Simples} o torneio organizado da seguinte forma:
\begin{itemize}[noitemsep]
	\item Todos as participantes se enfrentam seguindo o sistema de \textit{Disputa em Eliminatórias Simples}.
	\item A \textit{Fase Classificatória} é formada pelos confrontos de oitavas-de-final (O1 a O8) e de quartas-de-final (Q1 a Q4)
	\item A \textit{Fase Final} é formada pelos confrontos de semifinal (S1 e S2) e pelo confronto final (F1)
	\item As participantes são classificadas da seguinte forma:
	\begin{itemize}[noitemsep]
		\item É declarada Campeã a vencedora do confronto final F1, e Vice-campeã a outra participante do confronto final F1.
		\item É declarada 3a colocada o participante que tiver se classificado para um dos confrontos semi-finais (S1 ou S2) e disputado tal confronto contra a Campeã. É declarada 4a colocada o participante que tiver se classificado para um dos confrontos semi-finais (S1 ou S2) e disputado tal confronto contra a Vice-campeã.
		\item É declarada 5a colocada a participante que tiver se classificado para um confronto de quartas-de-final (Q1 a Q4) e disputado tal confronto contra a equipe melhor classificada a disputar um confronto de quartas-de-final. É declarada 6a colocada a participante que tiver se classificado para um confronto de quartas-de-final (Q1 a Q4) e disputado tal confronto contra a próxima equipe melhor classificada a disputar um confronto de quartas-de-final, e assim por diante até a 8a colocada, se houver
		\item É declarada 9a colocada a participante que tiver se classificado para um confronto de oitavas-de-final (O1 a O8) e disputado tal confronto contra a equipe melhor classificada a disputar um confronto de oitavas-de-final. É declarada 10a colocada a participante que tiver se classificado para um confronto de oitavas-de-final (O1 a O8) e disputado tal confronto contra a próxima equipe melhor classificada a disputar um confronto de oitavas-de-final, e assim por diante até a 16a colocada, se houver
		\item Caso uma ou mais participante(s) não dispute(m) seu(s) confronto(s) em determinada fase dos confrontos do sistema de \textit{Disputa em Eliminatórias Simples}, ela(s) será(ão) classificada(s) com a pior classificação dentre as participante que chegaram até aquela fase dos confrontos
	\end{itemize}
\end{itemize}

\noindent
Fica definido como \textit{Torneio de Grupos Mais Eliminação Simples} o torneio organizado da seguinte forma:
\begin{itemize}[noitemsep]
	\item A \textit{Fase Classificatória} é formada por disputados no sistema de \textit{Disputa em Grupos}.

	\item A \textit{Fase Final} é formada por disputados no sistema de \textit{Disputa em Eliminatórias Simples}, participando desta fase um número de participantes de acordo com a quantidade de participantes inscritas no torneio:
	\begin{itemize}[noitemsep]
		\item Havendo até 8 participantes inscritas, as 4 (quatro) participantes melhor classificadas na\textit {Fase Classificatória} disputam a \textit{Fase Final}, sendo consideradas as 4 (quatro) cabeças de chave desta fase de acordo a ordenação obtida na \textit{Disputa em Grupos}.
		\item Havendo entre 9 e 12 participantes inscritas, as 6 (seis) participantes melhor classificadas na\textit {Fase Classificatória} disputam a \textit{Fase Final}, sendo consideradas as 4 (quatro) cabeças de chave desta fase as 4 (quatro) participantes melhor classificadas na \textit{Disputa em Grupos}, 1a sorteada a 5a melhor classificada na \textit{Disputa em Grupos} e 2a sorteada a 6a melhor classificada na \textit{Disputa em Grupos}.
		\item Havendo entre 13 e 14 participantes inscritas, as 8 (oito) participantes melhor classificadas na \textit{Fase Classificatória} disputam a \textit{Fase Final}, sendo consideradas as 4 (quatro) cabeças de chave desta fase as 4 (quatro) participantes melhor classificadas na \textit{Disputa em Grupos}, 1a sorteada a 5a melhor classificada na \textit{Disputa em Grupos}, 2a sorteada a 6a melhor classificada na \textit{Disputa em Grupos} e assim por diante.
	\end{itemize}

	\item As participantes são classificadas da seguinte forma:
	\begin{itemize}[noitemsep]
		\item É declarada Campeã a vencedora do confronto final F1, e Vice-campeã a outra participante do confronto final F1.
		\item É declarada 3a colocada o participante que tiver se classificado para um dos confrontos semi-finais (S1 ou S2) e disputado tal confronto contra a Campeã. É declarada 4a colocada o participante que tiver se classificado para um dos confrontos semi-finais (S1 ou S2) e disputado tal confronto contra a Vice-campeã.
		\item Caso o número de participantes que disputaram a \textit{Fase Final} tenha sido maior que 4 (quatro), é declarada 5a colocada a participante que tiver sido eliminada em um confronto de quartas-de-final (Q1 a Q4) e tiver tido a melhor classificação na \textit{Fase Classificatória}; é declarada 6a colocada a participante que tiver sido eliminada em um confronto de quartas-de-final (Q1 a Q4) e tiver tido proxima melhor classificação na \textit{Fase Classificatória}, e assim por diante.
		\item Caso uma ou mais participante(s) não dispute(m) seu(s) confronto(s) em determinada fase dos confrontos do sistema de \textit{Disputa em Eliminatórias Simples}, ela(s) será(ão) classificada(s) com a pior classificação dentre as participante que chegaram até aquela fase dos confrontos
		\item As demais participantes serão classificadas de acordo com a ordenação obtida no sistema de\textit {Disputa em Grupos}.
	\end{itemize}
\end{itemize}

\noindent
Fica definido como \textit{Torneio por Provas} o torneio organizado da seguinte forma:
\begin{itemize}[noitemsep]
	\item Para cada modalidade, é definido um \textit{Conjunto de Provas}, com um peso atribuído para cada prova, um \textit{Sistema de Classificação}, com uma pontuação atribuída de acordo com cada classificação em uma prova, e um conjunto de \textit{Regras de Inscrição} para cada prova.

	\item As participantes competem em cada prova do \textit{Conjunto de Provas} com um número de atletas ou equipes que obedeça às \textit{Regras de Inscrição} \item Para cada atleta ou equipe é atribuída uma pontuação de acordo com o \textit{Sistema de Classificação} e as \textit{Regras de Inscrição}

	\item As pontuações obtidas por cada atleta ou equipe de uma participante em cada prova de acordo com o \textit{Sistema de Classificação} serão somadas para totalizar a pontuação da participante.

	\item As participantes são ordenadas de acordo com os seguintes critérios:
	\begin{itemize}[noitemsep]
		\item Pontuação total obtida;
		\item Número de primeiros lugares obtidos, depois número de segundos lugares obtidos e assim por diante, se necessário.
	\end{itemize}
\end{itemize}
