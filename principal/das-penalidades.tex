{\let\clearpage\relax \chapter{Das Penalidades}}

\begin{article}
	A entidade que abandonar as disputas ou não comparecer na \textbf{INTERCOMP} daquele ano, será expulsa da competição, cumprindo, pelo menos, 1 (uma) edição da competição de suspensão.

	\begin{xparagraph}
		A entidade expulsa será automaticamente a última colocada da competição.
	\end{xparagraph}

	\begin{xparagraph}
		Caracteriza-se abandono dos jogos o não comparecimento de 50\% ou mais de seus jogos. Multada em 2 salários mínimos
	\end{xparagraph}
\end{article}

\begin{article}
	A entidade que perder por W.O. em qualquer das partidas de qualquer modalidade perderá o equivalente à pontuação do primeiro colocado na contagem geral, além de não receber os pontos relativos à colocação em que se classifique, e também deverá pagar uma multa de 1/4 salário mínimo (baseado no salário mínimo vigente no momento do pagamento) até o final da competição. Nesse caso as entidades subsequentes não subirão na colocação final.

	\begin{xparagraph}
		Para todas as modalidades, excetuando-se o Xadrez que tem critérios próprios, o W.O. deverá ser aplicado pelo árbitro ou pelo representante da partida, 15 (quinze) minutos após o horário previamente estabelecido pela Comissão Técnica para o início da mesma, ou por qualquer situação prevista neste estatuto.
	\end{xparagraph}
\end{article}

\begin{article}
	A entidade que tiver um atleta não listado integrando alguma de suas equipes será automaticamente \textbf{desclassificada} da INTERCOMP corrente.

	\begin{xparagraph}
		Para participar novamente da INTERCOMP, a entidade deverá pagar a taxa de 1/2 salário mínimo (baseado no salário mínimo vigente no momento do pagamento) da edição da qual está fazendo a inscrição.
	\end{xparagraph}

	\begin{xparagraph}
		Caso a faculdade tenha sido vencedora de um jogo em que um atleta não listado integrou sua equipe, o adversário passa ser o vencedor do jogo.
	\end{xparagraph}
\end{article}

\begin{article}
	\label{art:infracoes}
	Serão aplicadas penas disciplinares (Advertência, Suspensão e/ou Eliminação dos Jogos) às pessoas e entidades inscritas em súmula, ou se não inscritas, pertencentes às diretorias das entidades inscritas que tenham incorrido nas seguintes infrações:

	\begin{enumerate}[noitemsep]
		\item \label{inf:inc.desrespeito}
			Incentivar os atletas ao desrespeito às autoridades; 1/4
		\item \label{inf:inc.violencia}
			Estimular os atletas à prática da violência; 1/4
		\item Atirar objetos dentro dos locais dos jogos; 1/4
		\item Invadir os locais dos jogos durante as partidas; 1/4
		\item \label{inf:agr.verbal}
			Agredir verbalmente e/ou moralmente árbitros, dirigentes da competição, demais autoridades e/ou adversários; 1/2
		\item Tentar ou agredir fisicamente árbitros, dirigentes da competição, demais autoridades e/ou adversários; 1/2
		\item Depredar as instalações ou locais de jogos; Valor da depredação ou 1/2 (maior)
		\item \label{inf:anti.desport}
			Praticar atitudes antidesportivas 1/4
	\end{enumerate}

	\begin{xparagraph}
		cada pena disciplinar será avaliada pela Comissão Organizadora.
	\end{xparagraph}
\end{article}

\begin{article}
	As pessoas não incluídas no art. \ref{art:infracoes}o que incorrerem nas infrações acima, exceto as alíneas \ref{inf:inc.desrespeito}, \ref{inf:inc.violencia}, \ref{inf:agr.verbal} e \ref{inf:anti.desport}, tendo sido identificadas em relatório elaborado pelo representante e/ou árbitro como integrantes da torcida da entidade, e tendo esta informação comprovada, também poderão ter penas disciplinares aplicadas. Esses casos também serão avaliados pela CO.
\end{article}

\begin{article}
	No caso de qualquer tipo de assédio, agressão, física ou não, de caráter discriminatório, baseada em gênero, raça, etnia, orientação sexual, religião, dentre outros, nas dependências da INTERCOMP, poderá implicar em punição individual ou institucional a depender da avaliação da CO.

	\begin{xparagraph}
		Nos casos de agressão fora de jogo, nas outras dependências do evento -- ginásio, balada, alojamento, etc. --, a denúncia deverá ser reportada à CO, que levará o caso para análise. A denúncia poderá ser feita pela vítima, seu(s) representante(s) ou testemunha (s) do ocorrido.
	\end{xparagraph}
\end{article}
