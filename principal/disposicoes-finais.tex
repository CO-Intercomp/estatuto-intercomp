{\let\clearpage\relax \chapter{Das Disposições Finais}}

\begin{article}
	A Comissão Organizadora da \textbf{INTERCOMP} reconhecerá como chefe da delegação o presidente da atlética ou a pessoa credenciada por ele.
\end{article}

\begin{article}
	O presente estatuto entrará em vigor logo após sua aprovação por um representante da diretoria das atléticas fundadoras.
\end{article}

\begin{article}
	Os casos omissos no presente estatuto deverão ser resolvidos pela Comissão Organizadora da INTERCOMP.
\end{article}

\begin{article}
	O regulamento específico das modalidades a seguir entrará em vigor logo após sua aprovação pelas diretorias das entidades participantes e fundadoras.
\end{article}

\begin{article}
	Após a apresentação do orçamento pela C.O. da \textbf{INTERCOMP} será cobrada das Faculdades participantes uma taxa para a realização dos jogos, cujo valor será fixado pela C.O. Taxa a qual deverá ser paga à Tesouraria na data estipulada.
\end{article}

\begin{article}
	As delegações concorrentes serão responsáveis pela boa conservação dos locais dos jogos no decorrer da competição, obrigando-se a acatar as ordens disciplinares dos mesmos e indenizar a C.O. pelos danos eventualmente verificados no material posto à disposição.
\end{article}

\begin{article}
	A \textbf{INTERCOMP} somente não se realizará em caso de força maior ou calamidade pública ou caso fortuito, caso a resolução seja aceita em reunião da Comissão Organizadora.
\end{article}

\begin{article}
	Ementas a este estatuto serão aceitas se e somente se contiver a assinatura dos representantes legais de \textbf{todas as faculdades integrantes do ano corrente}.
\end{article}

\begin{article}
	Revogam-se todas as disposições em contrário.
\end{article}

\begin{article}
    Qualquer caso de assédio e abuso será passível de expulsão imediata
\end{article}
