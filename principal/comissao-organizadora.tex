{\let\clearpage\relax \chapter{Comissão Organizadora}}

\begin{article}
	Serão nomeados, através de uma votação: um presidente, um vice-presidente, um representante de infraestrutura, um representante financeiro, um representante de marketing e um representante esportivo. Estes irão compor a Comissão Organizadora Executiva.
\end{article}

\begin{article}
	Os membros da Comissão Organizadora Executiva serão eleitos através de votação simples de todas as faculdades fundadoras ou participantes da \textbf{INTERCOMP}.

	\begin{xparagraph}
		As funções do presidente e do vice-presidente são:
		\begin{enumerate}[noitemsep]
			\item Dirigir as reuniões da Comissão Organizadora.
			\item Proclamar as decisões resolvidas pela Comissão Organizadora, através da ata de reunião, assinadas pelos membros da Comissão Organizadora Executiva.
		\end{enumerate}
	\end{xparagraph}

	\begin{xparagraph}
		As funções do setor financeiro são:
		\begin{enumerate}[noitemsep]
			\item Redigir um balanço financeiro ao final da competição, o qual deverá ser registrado pelo presidente e/ou vice-presidente.
			\item Usar, somente para tornar possível a \textbf{INTERCOMP} e os eventos oficialmente reconhecidos por ela, os recursos levantados pelas atléticas participantes de acordo com as resoluções da Comissão Organizadora, sujeito responsabilidades previstas pela lei. Ser responsável pela promoção de condições para que haja patrocínios (emissão de documentos, etc.).
		\end{enumerate}
	\end{xparagraph}

	\begin{xparagraph}
		As funções do representante de infraestrutura são:
		\begin{enumerate}[noitemsep]
			\item Averiguar condições estruturais, sociais, físicas e esportivas da cidade que receberá a \textbf{INTERCOMP}.
			\item Assegurar organização e conforto a todos os atletas que irão participar do evento.
		\end{enumerate}
	\end{xparagraph}

	\begin{xparagraph}
		As funções do representante esportivo são:
		\begin{enumerate}[noitemsep]
			\item Aprovar a participação e as restrições das faculdades em cada modalidade inscrita.
			\item Montar o chaveamento e as tabelas das modalidades cumprindo as exigências deste estatuto.
			\item Zelar pelo andamento esportivo no decorrer da competição.
			\item Reunir todas as súmulas durante a competição para avaliação da C.O.
		\end{enumerate}
	\end{xparagraph}

	\begin{xparagraph}
		Esta nomeação poderá ser alterada, antes da competição, por decisão da maioria absoluta.
	\end{xparagraph}

	\begin{xparagraph}
		Um membro da Comissão Organizadora Executiva que faltar duas vezes, sem justificativa aprovada pelos outros membros, durante as reuniões anteriores à \textbf{INTERCOMP} será automaticamente eliminado da Comissão Organizadora Executiva. A Comissão Organizadora irá escolher um novo representante para o cargo.
	\end{xparagraph}

	\begin{xparagraph}
		É de competência da Comissão Organizadora Executiva:
		\begin{enumerate}[noitemsep]
			\item Coordenar as reuniões da Comissão Organizadora;
			\item Coordenar o bom andamento dos jogos;
			\item No caso de contratada uma empresa para auxiliar a organização, em qualquer nível que seja coordenar, controlar e exigir o bom andamento dos jogos, assegurando o correto desenvolvimento do campeonato, conferindo a pontuação e atestando os campeões;
		\end{enumerate}
	\end{xparagraph}
\end{article}

\begin{article}
	A Comissão Organizadora da \textbf{INTERCOMP} deverá ser composta ao menos pelos presidentes das entidades integrantes, e por outros representantes à sua escolha.

	\begin{xparagraph}
		É de competência da Comissão Organizadora:
		\begin{enumerate}[noitemsep]
			\item Divulgar a competição;
			\item Definir a data para a realização da \textbf{INTERCOMP};
			\item Interpretar esse estatuto, e aplicar as penalidades previstas;
		\end{enumerate}
	\end{xparagraph}
\end{article}

\begin{article}
	A Comissão Organizadora tem plenos poderes para organizar a \textbf{INTERCOMP}.

	\begin{xparagraph}
		Para modificar este Estatuto será necessária a aprovação por maioria simples das entidades \textbf{fundadora} e da Comissão Organizadora Exexcutiva. Em caso de aprovação de metade das entidades, a Comissão Organizadora é quem decide se haverá a modificação.
	\end{xparagraph}

	\begin{xparagraph}
		A participação de um \textbf{integrante} que tenha cometido uma grave irregularidade que possa prejudicar o bom andamento da \textbf{INTERCOMP} será condicionada a uma aprovação por maioria absoluta (\sfrac{2}{3}) de cada atlética \textbf{fundadora} e da Comissão Organizadora Executiva. Em caso de aprovação de metade das entidades, a Comissão Organizadora é quem aprova, ou não, a participação.
	\end{xparagraph}
\end{article}

\begin{article}
	Todas as alterações do estatuto devem ser feitas através de reunião convocada pela C.O.

	\begin{xparagraph}
		As reuniões devem ter presença de todas as fundadoras.
	\end{xparagraph}

	\begin{xparagraph}
		As alterações serão aprovadas através de voto simples.
	\end{xparagraph}
\end{article}

\begin{article}
	As entidades \textbf{fundadoras} ou \textbf{participantes} que faltarem a duas reuniões (consecutivas ou alternadas) perderão o direito a voto da próxima reunião que comparecerem.

	\begin{xparagraph}
		As entidades \textbf{convidadas} não têm direito a voto.
	\end{xparagraph}

	\begin{xparagraph}
		Cada entidade \textbf{fundadora} ou \textbf{participante} tem direito no máximo a um voto em qualquer votação realizada pela Comissão Organizadora.
	\end{xparagraph}

	\begin{xparagraph}
		Será considerada uma \textit{perda de voto} em uma reunião para determinada entidade se não houver comparecimento de nenhum representante após os primeiros 30 (trinta) minutos de uma reunião.
	\end{xparagraph}
\end{article}

\begin{article}
	Todo recurso deverá ser apresentado por escrito à Comissão Organizadora, por um membro da atlética integrante interessada, no início dos trabalhos da próxima reunião.

	\begin{xparagraph}
		O recorrente poderá trazer uma única pessoa, estranha à Comissão Organizadora, para testemunhar.
	\end{xparagraph}

	\begin{xparagraph}
		O recorrente poderá também pedir o depoimento dos árbitros e/ou mesários dos jogos no decorrer da competição, sendo sua obrigação trazê-los pessoalmente à reunião da Comissão Organizadora.
	\end{xparagraph}

	\begin{xparagraph}
		Cada recurso só poderá ser deliberado uma única vez.
	\end{xparagraph}

	\begin{xparagraph}
		Só serão cabíveis os recursos que tratarem de assuntos omissos neste estatuto.
	\end{xparagraph}

	\begin{xparagraph}
		Não serão cabíveis recursos em qualquer outra instância que não a esportiva.
	\end{xparagraph}

	\begin{xparagraph}
		As entidades \textbf{integrantes envolvidas} no recurso não têm direito a voto.
	\end{xparagraph}

	\begin{xparagraph}
		A Comissão Organizadora deverá analisar os recursos técnicos.
	\end{xparagraph}

	\begin{xparagraph}
		Caso haja empate em uma votação, esta será submetida a uma nova apuração onde só participarão as entidades \textbf{fundadoras}; permanecendo o empate, o voto de Minerva será dada pela Comissão Organizadora Executiva.
	\end{xparagraph}
\end{article}

\begin{article}
	A Comissão Organizadora deverá se reunir no mínimo 6 (seis) meses antes da competição.

	\begin{xparagraph}
		É de responsabilidade da presidência as comunicações para as convocações das reuniões da Comissão Organizadora.
	\end{xparagraph}

	\begin{xparagraph}
		É necessário que ocorra no mínimo uma reunião até a data da \textbf{INTERCOMP}.
	\end{xparagraph}
\end{article}

\begin{article}
	A Comissão Organizadora deverá se reunir, durante a competição, todos os dias após o término dos jogos e, se necessário, durante o dia, em caráter extraordinário, para exercer suas funções.

	\begin{xparagraph}
		Faltas não serão consideradas na reunião do último dia de competição.
	\end{xparagraph}

	\begin{xparagraph}
		As reuniões extraordinárias são convocadas exclusivamente pela presidência.
	\end{xparagraph}
\end{article}

\begin{article}
	Não serão aceitas procurações em reuniões da \textbf{INTERCOMP}.
\end{article}

\begin{article}
	A Comissão Organizadora deverá escolher, dentre propostas apresentadas, qual será a cidade-sede da \textbf{INTERCOMP}. Para isto, deverão ser realizadas visitas a todas as cidades candidatas, e caberá à Comissão Organizadora avaliar as condições e adequações de cada cidade à \textbf{INTERCOMP}.

	\begin{xparagraph}
		Após a escolha da cidade-sede, cabe à Comissão Organizadora apresentar às atléticas participantes um orçamento por escrito da \textbf{INTERCOMP}.
	\end{xparagraph}

	\begin{xparagraph}
		A Comissão Organizadora Executiva deverá fixar um valor a ser cobrado de cada atlética participante para a viabilização da \textbf{INTERCOMP}. O valor por pacote deve ser o mesmo para todas as delegações.
	\end{xparagraph}

	\begin{xparagraph}
		Esta taxa deverá ser paga dentro de um prazo estipulado pela Comissão Organizadora.
	\end{xparagraph}

	\begin{xparagraph}
		A Comissão Organizadora Executiva se responsabilizará pelos alojamentos e eventuais danos causados neles. Isso será feito mediante negociação com o responsável pelos alojamentos.
	\end{xparagraph}

	\begin{xparagraph}
		Cada faculdade deverá deixar com a Comissão Organizadora Executiva um cheque calção no valor de R\$ 1.000,00 (um mil reais) para cobrir eventuais danos que a faculdade causar ao alojamento que ficar alojada.
	\end{xparagraph}

	\begin{xparagraph}
		Caso os danos sejam superiores ao valor do cheque calção, caberá a faculdade arcar com os restos dos custos.
	\end{xparagraph}
\end{article}

\begin{article}
	Em no máximo um mês após o término da \textbf{INTERCOMP}, a Comissão Organizadora deverá se reunir para a apresentação de um relatório técnico, apreciação do balanço financeiro apresentado pelo financeiro e apresentação de propostas de data e local da próxima \textbf{INTERCOMP}.

	\begin{xparagraph}
		Estas informações deverão ser encaminhadas às entidades \textbf{integrantes}.
	\end{xparagraph}

	\begin{xparagraph}
		Na primeira reunião após a realização do torneio, deverão ser apresentados 5 representantes por faculdade para participar das reuniões para a edição seguinte. Estes representantes poderão ser modificados mediante aprovação da Comissão Organizadora.
	\end{xparagraph}
\end{article}
