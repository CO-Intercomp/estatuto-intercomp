\chapter{Disposições Fundamentais}

\begin{article}
	Este estatuto trata das disposições que regem as competições da \textbf{INTERCOMP}, e é soberano à Comissão Organizadora durante a competição.
\end{article}

\begin{article}
	As entidades participantes da \textbf{INTERCOMP} são conhecedoras das leis desportivas nacionais e deste estatuto e assim se submeterão a todas as consequências que delas possam emanar.
\end{article}

\begin{article}
	\label{art:integrantes}
	Denomina-se \textbf{INTERCOMP} a competição poli-esportiva realizada anualmente entre alunos dos Institutos de Ensino Superior descritas no \textbf{Termo de Declaração de Participantes} vigente.

	\begin{xparagraph}
		São consideradas \textbf{integrantes} da \textbf{INTERCOMP} as entidades descritas no \textbf{Termo de Declaração de Participantes} vigente.
	\end{xparagraph}

	\begin{xparagraph}
		São consideradas \textbf{fundadoras} da \textbf{INTERCOMP} as entidades:
		\begin{itemize}[noitemsep,leftmargin=2\parindent]
			\item \textbf{USP - São Carlos}, representada pela SACIM, Secretaria Acadêmica da Computação, Informática e Matemática
			\item \textbf{UNICAMP - Campinas}, representada pela Liga CEM, Liga da Computação, Estatística e Matemática
			\item \textbf{UFSCar - São Carlos}, representada pela Associação Atlética Acadêmica Computação Federal
		\end{itemize}

		Caso haja divergência das entidades fundadoras listadas neste documento e no termo de declaração de participantes, prevalecerá o determinado neste documento.
	\end{xparagraph}

	\begin{xparagraph}
		Compõem \textbf{Comissão Organizadora Executiva} da \textbf{INTERCOMP} as entidades descritas no \textbf{Termo de Declaração de Participantes} vigente.
	\end{xparagraph}

	\begin{xparagraph}
		São consideradas \textbf{participantes} da \textbf{INTERCOMP} as entidades descritas no \textbf{Termo de Declaração de Participantes} vigente.
	\end{xparagraph}

	\begin{xparagraph}
		São consideradas \textbf{convidadas} da \textbf{INTERCOMP} as entidades descritas no \textbf{Termo de Declaração de Participantes} vigente.
	\end{xparagraph}

	\begin{xparagraph}
		Será permitida a participação de entidades \textbf{convidadas} do \textbf{INTERCOMP} que abranjam os seguintes cursos, aqui denominados \textit{estatutários}:
		\begin{enumerate}[noitemsep,leftmargin=2\parindent]
			\item Represente e participe somente com alunos dos cursos:
			\begin{itemize}[noitemsep]
				\item Análise de Sistemas
				\item Bacharelado em Ciência da Computação
				\item Bacharelado em Estatística
				\item Bacharelado em Gestão da Informação
				\item Bacharelado em Matemática
				\item Bacharelado em Matemática Aplicada
				\item Bacharelado em Sistemas de Informação
				\item Engenharia de Computação
				\item Gestão de Tecnologia
				\item Informática Biomédica
				\item Licenciatura em Matemática
				\item Mídias Digitais Informática
				\item Modelagem Computacional
				\item Tecnologia de Software
				\item Tecnologia em Processamento de Dados
				\item Tecnologia em Redes de Computadores
				\item Tecnologia em Sistemas da Informação
				\item Física Computacional
				\item Bacharelado em Física
				\item Bacharelado em Química
				\item Licenciatura em Física
				\item Licenciatura em Química
				\item Pós-Graduação em algum dos cursos acima
				\item Outros cursos que deverão ser aprovados pela Comissão Organizadora
			\end{itemize}

			\item Tais cursos devem ser reconhecidos pelo MEC
			\item Tenham a sua participação aprovada pela Comissão Organizadora vigente
		\end{enumerate}
	\end{xparagraph}

	\begin{xparagraph}
		O Instituto de Ensino Superior que possuir uma representação com alguma dívida pendente com a \textbf{INTERCOMP}, não poderá participar da mesma, a não ser que entre em negociação da dívida em tempo hábil estipulado pela Comissão Organizadora.
	\end{xparagraph}

	\begin{xparagraph}
		A entidade \textbf{convidada} que participar de \textbf{duas edições consecutivas} do \textbf{INTERCOMP} poderá pleitear um \textbf{cargo na CO Executiva} na edição seguinte.
	\end{xparagraph}

	\begin{xparagraph}
		A entidade \textbf{participante} que não participar da \textbf{INTERCOMP} por \textbf{duas edições}, consecutivas ou por \textbf{duas edições}, perderá automaticamente o status de \textbf{participante} e entrará como \textbf{convidada} na edição seguinte que participar.
	\end{xparagraph}

	\begin{xparagraph}
		A entidade \textbf{fundadora} que não participar da \textbf{INTERCOMP} por \textbf{duas edições} consecutivas ou por \textbf{duas edições} não consecutivas em um prazo de cinco anos, perderá automaticamente o status de \textbf{fundadora} e entrará como \textbf{convidada} na edição seguinte que participar, sendo automaticamente classificada como \textbf{participante} na segunda edição consecutiva de participação.
	\end{xparagraph}
\end{article}

\begin{article}
	Ao completarem-se dois anos de participação de uma entidade \textbf{convidada}, as entidades \textbf{integrantes} deverão votar sua efetivação ou remoção da \textbf{INTERCOMP}. É necessária maioria absoluta para remoção, dessa forma a entidade não participará da edição imediatamente seguinte. Caso contrário a entidade será automaticamente classificada como \textbf{participante} na edição seguinte. Uma entidade \textbf{convidada} que não foi efetivada pode voltar a ser convidada em edições futuras.
\end{article}

\begin{article}
	Qualquer faculdade que queira se inscrever na \textbf{INTERCOMP} deverá fazê-lo através de ofício, no prazo máximo estipulado pela Comissão Organizadora.
\end{article}

\begin{article}
	Estarão aptas a participar da \textbf{INTERCOMP} as entidades que efetuarem sua inscrição dentro do prazo estabelecido pela Comissão Organizadora, desde que atendam aos requisitos listados neste estatuto
\end{article}

\begin{article}
	Uma entidade \textbf{participante} será expulsa por votação das entidades \textbf{integrantes} se a maioria qualificada de 2/3 (dois terços) for a favor da expulsão. A entidade expulsa poderá voltar a participar de uma edição posterior caso a Comissão Organizadora julgue conveniente.
\end{article}
