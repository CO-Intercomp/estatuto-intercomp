\chapter{Disposições Fundamentais}

\begin{article}
	Este estatuto trata das disposições que regem as competições da \textbf{INTERCOMP}, e é soberano à Comissão Organizadora durante a competição.
\end{article}

\begin{article}
	As entidades participantes da \textbf{INTERCOMP} são conhecedoras das leis desportivas nacionais e deste estatuto e assim se submeterão a todas as consequências que delas possam emanar.
\end{article}

\begin{article}
	\label{art:integrantes}
	Denomina-se \textbf{INTERCOMP} a competição poli-esportiva realizada anualmente entre alunos dos seguintes Institutos de Ensino Superior:
	\begin{description}[noitemsep]
		\item[USP - São Carlos] representada pela SACIM -- Secretaria Acadêmica da Computação, Informática e Matemática --, com alunos dos cursos:
		\begin{itemize}[noitemsep]
			\item Bacharelado em Ciência da Computação
			\item Bacharelado em Estatística
			\item Bacharelado em Informática
			\item Bacharelado em Matemática
			\item Bacharelado em Matemática Aplicada e Computação Científica
			\item Bacharelado em Sistemas de Informação
			\item Engenharia de Computação
			\item Física Computacional \textbf{\textit{(não estatutário)}}
			\item Licenciatura em Ciências Exatas \textbf{\textit{(não estatutário)}}
			\item Licenciatura em Matemática
			\item Pós-Graduação em Ciências da Computação
			\item Pós-Graduação em Matemática
		\end{itemize}

		\item[UNICAMP - Campinas] representada pela Liga CEM -- Liga da Computação, Estatística e Matemática --, com alunos dos cursos:
		\begin{itemize}[noitemsep]
			\item Bacharelado em Ciência da Computação
			\item Bacharelado em Estatística
			\item Bacharelado em Matemática
			\item Bacharelado em Matemática Aplicada e Computacional
			\item Básico Integrado em Matemática, Física e Matemática Aplicada e Computacional
			\item Engenharia de Computação
			\item Licenciatura em Matemática
			\item Pós-Graduação em Ciência da Computação
			\item Pós-Graduação em Estatística
			\item Pós-Graduação em Matemática
		\end{itemize}

		\item[UFSCar] representada pela AAACF -- Associação Atlética Acadêmica da Computação Federal --, com alunos dos cursos:
		\begin{itemize}[noitemsep]
			\item Bacharelado em Ciência da Computação
			\item Bacharelado em Estatística
			\item Bacharelado em Física \textbf{\textit{(não estatutário)}}
			\item Bacharelado em Matemática
			\item Engenharia de Computação
			\item Engenharia Física \textbf{\textit{(não estatutário)}}
			\item Licenciatura em Física \textbf{\textit{(não estatutário)}}
			\item Licenciatura em Matemática
			\item Pós-Graduação em Computação
			\item Pós-Graduação em Estatística
			\item Pós-Graduação em Matemática
		\end{itemize}

		\item[USP Leste - EACH] representada pelo DASI -- Diretório Acadêmico de Sistemas de Informação --, com alunos dos cursos:
		\begin{itemize}[noitemsep]
			\item Bacharelado em Gestão Ambiental \textbf{\textit{(não estatutário)}}
			\item Bacharelado em Marketing \textbf{\textit{(não estatutário)}}
			\item Bacharelado em Sistemas de Informação
			\item Pós-Graduação em Sistemas de Informação
		\end{itemize}

		\item[UEM] representada pela AAACEX -- Associação Atlética Acadêmica do Centro de Exatas --, com alunos dos cursos:
		\begin{itemize}[noitemsep]
			\item Bacharelado em Ciência da Computação
			\item Bacharelado em Estatística
			\item Bacharelado em Física \textbf{\textit{(não estatutário)}}
			\item Bacharelado em Informática
			\item Bacharelado em Matemática
			\item Bacharelado em Química \textbf{\textit{(não estatutário)}}
			\item Licenciatura em Física \textbf{\textit{(não estatutário)}}
			\item Licenciatura em Matemática
			\item Licenciatura em Química \textbf{\textit{(não estatutário)}}
			\item Pós-Graduação em Bioestatística \textbf{\textit{(não estatutário)}}
			\item Pós-Graduação em Ciência da Computação
			\item Pós-Graduação em Física \textbf{\textit{(não estatutário)}}
			\item Pós-Graduação em Matemática
			\item Pós-Graduação em Química \textbf{\textit{(não estatutário)}}
		\end{itemize}

		\item[Faculdade de Tecnologia - Carapicuíba] representada pela Atlética Fatec-Caracas, com alunos dos cursos:
		\begin{itemize}[noitemsep]
			\item Bacharelado em Tecnologia da Informação
			\item Tecnologia em Análise e Desenvolvimento de Sistemas
			\item Tecnologia em Jogos Digitais
			\item Tecnologia em Logística \textbf{\textit{(não estatutário)}}
			\item Tecnologia em Secretariado \textbf{\textit{(não estatutário)}}
			\item Tecnologia em Segurança da Informação \textbf{\textit{(não estatutário)}}
			\item Tecnologia em Sistemas para Internet
		\end{itemize}

		\item[UFJF] representada pela Associação Atlética Atlética ICE UFJF, com alunos dos cursos:
		\begin{itemize}[noitemsep]
			\item Bacharelado em Ciência da Computação
			\item Bacharelado em Ciências Exatas \textbf{\textit{(não estatutário)}}
			\item Bacharelado em Estatística
			\item Bacharelado em Física \textbf{\textit{(não estatutário)}}
			\item Bacharelado em Matemática
			\item Bacharelado em Química \textbf{\textit{(não estatutário)}}
			\item Bacharelado em Sistemas de Informação
			\item Engenharia Computacional
			\item Licenciatura em Física \textbf{\textit{(não estatutário)}}
			\item Licenciatura em Matemática
			\item Licenciatura em Química \textbf{\textit{(não estatutário)}}
			\item Pós-Graduação em Ciência da Computação
			\item Pós-Graduação em Física \textbf{\textit{(não estatutário)}}
			\item Pós-Graduação em Matemática
			\item Pós-Graduação em Modelagem Computacional \textbf{\textit{(não estatutário)}}
			\item Pós-Graduação em Química \textbf{\textit{(não estatutário)}}
		\end{itemize}

		\item[Universidade São Judas Tadeu] representada pela AAATIUSJT -- Associação Atlética Acadêmica de Tecnologia da Informação USJT --, com alunos dos cursos:
		\begin{itemize}[noitemsep]
			\item Análise de Sistemas
			\item Ciência da Computação
			\item Exatas
			\item Sistemas da Computação
		\end{itemize}
	\end{description}

	\begin{xparagraph}
		Estas entidades são consideradas \textbf{integrantes} da \textbf{INTERCOMP}.
	\end{xparagraph}

	\begin{xparagraph}
		São consideradas \textbf{fundadoras} da \textbf{INTERCOMP} as entidades:
		\begin{itemize}[noitemsep,leftmargin=2\parindent]
			\item USP - São Carlos
			\item UNICAMP - Campinas
			\item UFSCar
		\end{itemize}
	\end{xparagraph}

	\begin{xparagraph}
		São consideradas \textbf{participantes} da \textbf{INTERCOMP} as entidades:
		\begin{itemize}[noitemsep,leftmargin=2\parindent]
			\item USP Leste - EACH
			\item UEM
			\item Faculdade de Tecnologia - Carapicuíba
			\item UFJF
			\item USJT
		\end{itemize}
	\end{xparagraph}

	\begin{xparagraph}
		Esta edição não contará com a presença de entidades \textbf{convidadas} da \textbf{INTERCOMP}.
	\end{xparagraph}

	\begin{xparagraph}
		Será permitida a participação de entidades \textbf{convidadas} do \textbf{INTERCOMP} que abranjam os seguintes cursos, aqui denominados \textit{estatutários}:
		\begin{enumerate}[noitemsep,leftmargin=2\parindent]
			\item Represente e participe somente com alunos dos cursos:
			\begin{itemize}[noitemsep]
				\item Análise de Sistemas
				\item Bacharelado em Ciência da Computação
				\item Bacharelado em Estatística
				\item Bacharelado em Gestão da Informação
				\item Bacharelado em Matemática
				\item Bacharelado em Matemática Aplicada
				\item Bacharelado em Sistemas de Informação
				\item Engenharia de Computação
				\item Gestão de Tecnologia
				\item Informática Biomédica
				\item Licenciatura em Matemática
				\item Mídias Digitais Informática
				\item Modelagem Computacional
				\item Tecnologia de Software
				\item Tecnologia em Processamento de Dados
				\item Tecnologia em Redes de Computadores
				\item Tecnologia em Sistemas da Informação
				\item Pós-Graduação em algum dos cursos acima
				\item Outros cursos que deverão ser aprovados pela Comissão Organizadora
			\end{itemize}

			\item Tais cursos devem ser reconhecidos pelo MEC
			\item Tenham a sua participação aprovada pela Comissão Organizadora vigente
		\end{enumerate}
	\end{xparagraph}

	\begin{xparagraph}
		O Instituto de Ensino Superior que possuir uma representação com alguma dívida pendente com a \textbf{INTERCOMP}, não poderá participar da mesma, a não ser que entre em negociação da dívida em tempo hábil estipulado pela Comissão Organizadora.
	\end{xparagraph}

	\begin{xparagraph}
		A entidade \textbf{fundadora} que não participar da \textbf{INTERCOMP} por \textbf{duas edições} consecutivas ou por \textbf{duas edições} não consecutivas em um prazo de cinco anos, perderá automaticamente o status de \textbf{fundadora} e entrará como \textbf{convidada} na edição seguinte que participar, sendo classificada como \textbf{participante} se efetivada na segunda edição consecutiva de participação.
	\end{xparagraph}

	\begin{xparagraph}
		A entidade \textbf{participante} que não participar da \textbf{INTERCOMP} por \textbf{duas edições}, consecutivas ou não, perderá automaticamente o status de \textbf{participante} e entrará como \textbf{convidada} na edição seguinte que participar.
	\end{xparagraph}
\end{article}

\begin{article}
	Ao completarem-se dois anos de participação de uma entidade \textbf{convidada}, as entidades \textbf{participantes} deverão votar sua efetivação ou remoção da \textbf{INTERCOMP}. É necessária maioria absoluta para remoção, dessa forma a entidade não participará da edição imediatamente seguinte. Caso contrário a entidade será automaticamente classificada como \textbf{participante} na edição seguinte. Uma entidade \textbf{convidada} que não foi efetivada pode voltar a ser convidada em edições futuras.
\end{article}

\begin{article}
	Qualquer faculdade que queira se inscrever na \textbf{INTERCOMP} deverá fazê-lo através de ofício, no prazo máximo estipulado pela Comissão Organizadora.
\end{article}

\begin{article}
	Estarão aptas a participar da \textbf{INTERCOMP} as entidades que efetuarem sua inscrição dentro do prazo estabelecido pela Comissão Organizadora, desde que atendam aos requisitos listados neste estatuto
\end{article}
