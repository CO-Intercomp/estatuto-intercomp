{\let\clearpage\relax \chapter{Competições}}

\begin{article}
	Os jogos e as competições terão início no horário fixado, sendo considerada perdedora por W.O. a entidade que não comparecer devidamente uniformizada (observadas as tolerâncias deste regulamento - Capítulo \ref{chp:material}), no local da competição dentro deste horário previsto, com tolerância máxima de 15 minutos a partir da abertura da súmula.
\end{article}

\begin{article}
	Somente a C.O. Executiva poderá transferir, em caso de extrema necessidade ou urgência, os jogos ou competições, obrigando-se, no entanto a comunicar através de uma reunião às atléticas participantes, possibilitando que essas se apresentem em local e horário determinado em condições de jogo ou de competição.
\end{article}

\begin{article}
	\label{art:jogos.interromp}
	Se a disputa de uma modalidade for interrompida por distúrbios provocados pela torcida, a disputa deverá continuar com portões fechados, após a retirada da torcida.

	\begin{xparagraph}
		Em jogos de quadra a continuidade da partida seria feita após o último jogo da noite com portão fechado, e em jogos de campo a continuidade da partida seria feita logo após a retirada da torcida.
	\end{xparagraph}

	\begin{xparagraph}
		Na continuação, prevalecerá a contagem e o tempo decorrido até o instante da paralisação, determinados pela arbitragem.
	\end{xparagraph}

	\begin{xparagraph}
		Poderão assistir aos jogos com portão fechado dois representantes de cada entidade \textbf{integrante}, bem como qualquer membro da Comissão Organizadora.
	\end{xparagraph}

	\begin{xparagraph}
		Este artigo é soberano às regras das federações
	\end{xparagraph}
\end{article}

\begin{article}
	Quaisquer jogos ou competições que venham a ser realizados terão os horários marcados no período de disputa da competição em questão.
\end{article}
