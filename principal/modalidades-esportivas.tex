{\let\clearpage\relax \chapter{Modalidades Esportivas}}

\begin{article}
	A competição se constituirá das seguintes modalidades, que serão disputadas em torneios com a seguinte configuração:

	\noindent
	\begin{table}[!h]
		\center
		\makegapedcells
		\begin{tabularx}{\linewidth}{
			|>{\hsize=.75\hsize}Y|
			>{\hsize=1\hsize}Y|
			>{\hsize=1.25\hsize}Y|
		}
			\hline
			BASQUETEBOL &MASCULINO E FEMININO &\textit{Torneio de Eliminação Simples}\\\hline
			FUTEBOL &MASCULINO &\textit{Torneio de Eliminação Simples}\\\hline
			FUTSAL &MASCULINO E FEMININO &\textit{Torneio de Eliminação Simples}\\\hline
			HANDEBOL &MASCULINO E FEMININO &\textit{Torneio de Eliminação Simples}\\\hline
			NATAÇÃO &MASCULINO E FEMININO &\textit{Torneio por Provas}\\\hline
			TÊNIS &MASCULINO E FEMININO &\textit{Torneio de Eliminação Simples}\\\hline
			TÊNIS DE MESA &MASCULINO E FEMININO &\textit{Torneio de Eliminação Simples}\\\hline
			TRUCO &MASCULINO E FEMININO &\textit{Torneio de Eliminação Simples}\\\hline
			VOLEIBOL &MASCULINO E FEMININO &\textit{Torneio de Eliminação Simples}\\\hline
			VOLEIBOL DE AREIA DE DUPLA &MASCULINO E FEMININO &\textit{Torneio de Eliminação Simples}\\\hline
			XADREZ &ABSOLUTO &\textit{Torneio de Grupos Mais Eliminação Simples}\\\hline
		\end{tabularx}

		\caption{Tabela de Modalidades}
	\end{table}
\end{article}

\begin{article}
	Todas as modalidades esportivas obedecerão às regras de suas respectivas federações paulistas, salvo disposições em contrário determinadas por este estatuto.
\end{article}

\begin{article}
	A participação de uma faculdade em cada modalidade é opcional. As participantes deverão inscrever-se nas modalidades em disputa até um prazo máximo a ser definido pela C.O. Cada participante poderá se inscrever em quantas modalidades desejar, não havendo um mínimo nem máximo.
\end{article}

\begin{article}
	As inscrições dos atletas poderão ser feitas com o jogo já iniciado, em todos os esportes, com exceção do Xadrez, Tênis de Mesa, Tênis e de partidas interrompidas, conforme art. \ref{art:jogos.interromp}\ulo, sendo este artigo soberano às regras das federações.
\end{article}

\begin{article}
	Qualquer faculdade que queira acrescentar uma nova modalidade à \textbf{INTERCOMP} deverá fazê-lo através de ofício, no prazo máximo estipulado pela Comissão Organizadora.

	\begin{xparagraph}
		A nova modalidade será avaliada pela C.O. e, em caso de aprovação, será disputada por um período de análise de dois anos, sem contar na pontuação geral.
	\end{xparagraph}

	\begin{xparagraph}
		Ao final dos dois anos haverá uma análise do quanto a nova modalidade agregou à competição como um todo. Caberá à Comissão Organizadora vigente decidir se a modalidade passará a contar na pontuação geral ou deixará de ser disputada.
	\end{xparagraph}
\end{article}
