{\let\clearpage\relax \chapter{Do Material Esportivo e Uniforme %
\label{chp:material}}}


\begin{article}
	As equipes deverão apresentar-se para a comissão de arbitragem devidamente uniformizados, conforme as regras determinadas para cada modalidade

	\begin{xparagraph}
		Nas modalidades coletivas de quadra, é de responsabilidade de cada uma das equipes apresentar, antes do confronto, uma bola em condições ideais de uso. Caso não seja apresentada nenhuma bola em condições de uso para o início da partida, ambas as equipes serão eliminadas por WO.
	\end{xparagraph}

	\begin{xparagraph}
		Nas modalidades coletivas, a equipe deve apresentar-se com camisas idênticas, sendo permitida mas desincentivada a colocação de fitas adesivas na camisa para alterar ou confeccionar a numeração da mesma, e deverão usar calça legging (exceto o goleiro) ou shorts adequados para a prática da modalidade. Nas modalidades futsal e futebol é obrigatório também o uso de meião e o uso de caneleira é opcional.
	\end{xparagraph}

	\begin{xparagraph}
		A interpretação do disposto no parágrafo acima e da possibilidade de sua aplicação em face da situação, é de responsabilidade única e exclusiva do representante da partida.
	\end{xparagraph}

	\begin{xparagraph}
		Caso seja verificada a participação de atleta em desacordo com o disposto no parágrafo segundo supra durante a disputa, o mesmo deverá deixar o local do confronto até sanar a irregularidade. Caso isto seja constatado somente após o encerramento da partida, não caberá protesto contra a não utilização do uniforme correto.
	\end{xparagraph}

	\begin{xparagraph}
		Nos casos dos confrontos em que ambas as equipes apresentem uniformes semelhantes, é de responsabilidade da equipe não mandante providenciar novo uniforme.
	\end{xparagraph}

	\begin{xparagraph}
		Fica determinado que a equipe mandante é sempre a equipe que estiver acima no chaveamento.
	\end{xparagraph}

	\begin{xparagraph}
		É permitida a utilização de coletes para a distinção dos uniformes, estes precisam ser numerados, em último caso, a numeração nos coletes pode ser apresentada de maneira improvisada, contanto que não se desfaça durante a partida.
	\end{xparagraph}
\end{article}
