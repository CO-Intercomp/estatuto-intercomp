{\let\clearpage\relax \chapter{Basquetebol}}

\begin{article}
	A modalidade será disputada nas categorias feminina e masculina.
\end{article}

\begin{article}
	Durante os jogos serão obedecidas as regras oficiais, ressalvando os dispostos nos demais artigos deste regulamento.
\end{article}

\begin{article}
	A categoria masculina será realizada em 4 (quatro) períodos de 10 (dez) minutos corridos, sendo os 2 (dois) últimos minutos do segundo e quarto período cronometrados, com 5 (cinco) minutos de descanso, entre o segundo e terceiro quarto; a categoria feminina será realizada em 4 (quatro) períodos de 8 (oito) minutos corridos, sendo os 2 (dois) últimos minutos do segundo e quarto período cronometrados, com 5 (cinco) minutos de descanso, entre o segundo e terceiro quarto.

	\begin{xparagraph}
		O cronômetro deverá ser travado nos lances livres, acidentes e pedidos de tempo.
	\end{xparagraph}

	\begin{xparagraph}
		Em caso de empate, ter-se-á prorrogação de 5 (cinco) minutos corridos, sendo o último minuto cronometrado. Persistindo o empate, serão efetuadas quantas prorrogações sejam necessárias para se determinar o vencedor.
	\end{xparagraph}
\end{article}

\begin{article}
	Cada entidade poderá inscrever na partida de basquetebol no máximo 12 (doze) atletas por partida.
\end{article}

\begin{article}
	Para cada entidade, dentre os jogadores inscritos em súmula apenas 2 (dois) podem estar formados e/ou ser alunos de um de seus cursos não estatutários.
\end{article}
