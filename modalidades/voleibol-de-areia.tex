{\let\clearpage\relax \chapter{Voleibol de Areia de Dupla}}

\begin{article}
	A modalidade será disputada nas categorias feminina e masculina.
\end{article}

\begin{article}
	Durante os jogos serão obedecidas as regras oficiais, ressalvando os dispostos nos demais artigos deste regulamento.
\end{article}

\begin{article}
	Cada entidade participante poderá inscrever no máximo 3 (três) atletas por partida de voleibol de areia, devendo indicar 2 (dois) deles como titulares e 1 (um) como reserva.

	\begin{xparagraph}
		O atleta reserva poderá substituir um dos atletas titulares após o fim de qualquer ponto, não podendo esta substituição ser desfeita até o fim do confronto.
	\end{xparagraph}
	
	\begin{xparagraph}
	    Para cada entidade, dentre os jogadores inscritos em súmula apenas 1 (um) pode estar formado e/ou ser aluno de um de seus cursos não estatutários.
	\end{xparagraph}
\end{article}

\begin{article}
	Cada partida de voleibol de areia será disputada em 3 (três) sets, sendo os 2 (dois) primeiros de 15 (quinze) pontos e o último de 11 (onze) pontos, se necessário. A partida final será disputada em 3 (três) sets, sendo os 2 (dois) primeiros de 21 (vinte e um) pontos e o último de 15 (quinze) pontos, se necessário.

	\begin{xparagraph}
		O set só termina quando um dos participantes obtiver uma vantagem de ao menos 2 (dois) pontos sobre o adversário
	\end{xparagraph}
\end{article}
