{\let\clearpage\relax \chapter{Voleibol}}

\begin{article}
	A modalidade será disputada nas categorias feminina e masculina.
\end{article}

\begin{article}
	Durante os jogos serão obedecidas as regras oficiais adotadas pela Confederação Brasileira de Voleibol (CBV), ressalvando os dispostos nos demais artigos deste regulamento.
\end{article}

\begin{article}
	As partidas serão disputadas em melhor de 3 (três) sets, sendo os 2 (dois) primeiros sets de 25 (vinte e cinco) pontos cada e o terceiro set, se necessário, de 15 (quinze) pontos; cada set só termina quando houver pelo menos 2 (dois) pontos de vantagem para uma equipe.

	\begin{xparagraph}
		No confronto final, a partida será disputada em melhor de 5 (cinco) sets, sendo os 4 (quatro) primeiros sets de 25 (vinte e cinco) pontos cada e o quinto set, se necessário, de 15 (quinze) pontos; cada set só termina quando houver pelo menos 2 (dois) pontos de vantagem para um equipe.
	\end{xparagraph}
\end{article}

\begin{article}
	Cada entidade poderá inscrever no máximo 14 (quatorze) atletas por partida, sendo que apenas 3 (três) podem ter se formado em 2018 ou 2019 e/ou ser alunos de um de seus cursos não estatutários.
\end{article}
