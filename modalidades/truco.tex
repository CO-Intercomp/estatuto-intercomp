{\let\clearpage\relax \chapter{Truco}}

\begin{article}
	A modalidade será disputada nas categorias feminina e masculina.
\end{article}

\begin{article}
	Cada entidade deverá apresentar para o representante da competição um baralho por dupla completo lacrado antes do primeiro confronto da competição.

	\begin{xparagraph}
		Ficará a cargo do árbitro a troca de baralho.
	\end{xparagraph}
\end{article}

\begin{article}
	Cada entidade poderá inscrever uma única dupla por categoria do truco.
\end{article}

\begin{article}
	Cada entidade deverá apresentar um fiscal que atuará em jogos que não envolva a sua agremiação.

	\begin{xparagraph}
		O fiscal não poderá opinar sobre a partida e só deve decidir para quem vai os tentos da mão em questão.
	\end{xparagraph}
\end{article}

\begin{article}
<<<<<<< HEAD
	Os confrontos serão disputados em melhor de 3 (três) partidas de 12 (doze) tentos.
\end{article}

\begin{article}
	Fica definido como \textit{pé} o jogador que está dando as cartas, como \textit{mão}, o jogador à direita do \textit{pé} e como \textit{cortador}, o jogador à esquerda do \textit{pé}.

	\begin{xparagraph}
		Somente após uma partida poderá um jogador trocar de lugar com o parceiro.
	\end{xparagraph}
\end{article}

\begin{article}
	O jogo será disputado com baralho sujo, ou seja, apenas sem as cartas 8 (oito), 9 (nove) e 10 (dez) e com ponto acima, ou seja, a manilha é a carta acima da carta virada.
\end{article}

\begin{article}
	O \textit{pé} receberá as cartas e deve as embaralhar livremente, passando-as ao \textit{cortador} antes de distribuí-las.

	\begin{xparagraph}
		O \textit{cortador} não poderá embaralhar as cartas.
	\end{xparagraph}

	\begin{xparagraph}
		Quando o corte for seco, o \textit{cortador} determinará se as cartas devem ser distribuídas por cima ou por baixo. Caso contrário, a escolha fica a cargo do \textit{pé}.
	\end{xparagraph}

	\begin{xparagraph}
		A primeira carta distribuída deve ser dada à \textit{mão}.
	\end{xparagraph}
\end{article}

\begin{article}
	Não será permitido em hipótese alguma ao \textit{pé} ou ao \textit{cortador} ver a frente ou as costas do baralho.

	\begin{xparagraph}
		Caso isso ocorra, o infrator perderá o tento e passa-se o baralho adiante.
	\end{xparagraph}
\end{article}

\begin{article}
	As cartas descartadas encobertas na segunda e terceira vasa não poderão ser vistas.
\end{article}

\begin{article}
	Somente por sinais os jogadores da mesma dupla poderão comunicar-se em relação ao jogo.

	\begin{xparagraph}
		Fica terminantemente proibido aos jogadores marcar as cartas com objetos, unhas, tintas, etc. ou utilizar objetos que possibilitem o reflexo das cartas.
	\end{xparagraph}
\end{article}

\begin{article}
	Havendo empate na primeira vasa, ninguém é obrigado a mostrar sua carta maior na segunda, mesmo com trucada, podendo a mão terminar na terceira vasa, valendo pois essa carta maior na jogada. Em caso de empate nas três vasas, sem trucada, ninguém ganha tento, passando-se o maço para frente.

	\begin{xparagraph}
		Quem truca ou retruca a carta exposta, perde em caso de empate.
	\end{xparagraph}

	\begin{xparagraph}
		Quem truca ou retruca no escuro, joga pelo empate.
	\end{xparagraph}
\end{article}

\begin{article}
	Na mão de onze, os dois jogadores poderão trocar suas cartas para conhecimento do jogo e depois decidirem se querem jogar ou não, cabendo a um deles expressar a decisão com as seguintes palavras: "vamos jogar" ou "não vamos jogar".

	\begin{xparagraph}
		Caso a dupla decida não jogar, os adversários ganham 1 (um) tento e passa-se o baralho adiante.
	\end{xparagraph}

	\begin{xparagraph}
		Caso a dupla decida jogar, esta deve ganhar a mão para ganhar a partida. Em caso de derrota ou empate, os adversários ganham 3 (três) tentos.
	\end{xparagraph}

	\begin{xparagraph}
		Caso ambas duplas estejam em mão de onze e haja um empate, deve-se passar o baralho adiante e jogar mais uma mão.
	\end{xparagraph}

	\begin{xparagraph}
		Quem truca na mão de onze perde a partida.
	\end{xparagraph}
=======
	As partidas serão disputadas em melhor de 03 (três) partidas de 12 (doze) pontos.
\end{article}

\begin{article}
	As partidas seguirão o seguinte regulamento:

	\textbf{REGULAMENTO DE JOGO}
	\begin{enumerate}[noitemsep]
		\item \textbf{DO BARALHO:}
		\begin{enumerate}[noitemsep]
			\item O jogo será disputado com baralho sujo, ou seja, sem as cartas 8 (oito), 9 (nove) e 10 (dez) e com ponto acima, ou seja, a manilha é a carta acima da carta virada.
			\item Ficará a cargo da comissão organizadora a troca de baralho quando necessária.
		\end{enumerate}

		\item \textbf{DA DADA DE CARTAS:}
		\begin{enumerate}[noitemsep]
			\item O jogador responsável de dar o baralho receberá as cartas e embaralhará as mesmas livremente e dará o baralho para o corte.
			\item É obrigatório que a primeira carta seja do jogador que esteja à sua direita (mão).
			\item Não será permitido em hipótese alguma ao pé (jogador que está dando as cartas) e ao cortador ver a frente ou as costas do baralho, e, caso isso ocorra, o infrator perderá o tento (ponto), sendo que, em qualquer dos casos, passa-se o baralho adiante.
			\item As cartas descartadas encobertas na segunda e terceira mão não poderão ser vistas.
			\item Somente após uma partida poderá um jogador trocar de lugar com o parceiro.
		\end{enumerate}

		\item \textbf{DO CORTE}
		\begin{enumerate}[noitemsep]
			\item O encarregado do corte não poderá embaralhar as cartas.
			\item O pé do baralho após o corte poderá dar as cartas por cima ou por baixo.
			\item Quando o corte for seco, o cortador determinará se o baralho será dado por cima ou por baixo.
		\end{enumerate}

		\item \textbf{DA CONVERSA ENTRE OS JOGADORES}
		\begin{enumerate}[noitemsep]
			\item Somente por sinais os jogadores da mesma dupla poderão comunicar-se em relação ao jogo.
			\item Fica terminantemente proibido aos jogadores: A - marcar as cartas com objetos, unhas, tintas, etc. B - utilizar objetos que possibilitem o reflexo das cartas.
		\end{enumerate}

		\item \textbf{DO EMPATE NA MÃO}
		\begin{enumerate}[noitemsep]
			\item Havendo empate na primeira vasa, ninguém é obrigado a mostrar sua carta maior na segunda, mesmo com trucada, podendo a mão terminar na terceira vasa, valendo pois essa carta maior na jogada. Em caso de empate nas três vasas, sem trucada, ninguém ganha tento, passando-se o maço para frente.
			\item Observando que:
			\begin{enumerate}[noitemsep]
				\item Quem truca ou retruca a carta exposta, perde em caso de empate.
				\item Quem truca ou retruca no escuro, joga pelo empate.
			\end{enumerate}
		\end{enumerate}

		\item \textbf{DA MÃO DE ONZE}
		\begin{enumerate}[noitemsep]
			\item Todas as partidas serão em doze pontos.
			\begin{enumerate}[noitemsep]
				\item Quando for mão de onze (escolha), para uma dupla não haverá empate, ou seja, se as três mãos terminarem empatadas, quem está com onze perde três tentos.
				\item Se uma das duplas estiver com onze tentos, quem mandou jogar, a que tem onze, deverá ganhar a jogada, pois se não o fizer perderá os três tentos.
				\item Se as duas estiverem com onze e o jogo terminar empatado, haverá necessidade de outra dada de cartas, passando-se, portanto, o baralho para o jogador seguinte.
			\end{enumerate}
			\item Na mão de onze, os dois jogadores poderão trocar suas cartas para conhecimento do jogo e depois resolverem, cabendo um deles determinar se jogam ou não, com as seguintes palavras: "vamos jogar", "não vamos jogar".
			\item Quem trucar na mão de onze perde a partida.
		\end{enumerate}
	\end{enumerate}
>>>>>>> master
\end{article}
