{\let\clearpage\relax \chapter{Futsal}}

\begin{article}
	A modalidade será disputada nas categorias feminina e masculina.
\end{article}

\begin{article}
	Durante os jogos serão obedecidas as regras oficiais adotadas pela Confederação Brasileira de Futebol de Salão (CBFS), ressalvando os dispostos nos demais artigos deste regulamento.
\end{article}

\begin{article}
	A modalidade será realizada em 2 (dois) períodos de 20 (vinte) minutos corridos, sendo os 2 (dois) últimos minutos de cada período cronometrados, com 5 (cinco) minutos de intervalo.

	\begin{xparagraph}
		Em caso de empate serão cobrados 5 (cinco) pênaltis alternadamente, por 5 (cinco) jogadores diferentes. Persistindo o empate, serão cobrados quantos pênaltis alternados forem necessários, até que se defina o vencedor, podendo se repetir os jogadores, após todos os jogadores terem cobrado.
	\end{xparagraph}

	\begin{xparagraph}
		Poderão cobrar as penalidades quaisquer jogadores que estiverem inscritos na súmula de jogo e não tenham sido expulsos, desqualificados ou excluídos da partida.
	\end{xparagraph}

	\begin{xparagraph}
		Em caso de empate na final, haverá uma prorrogação de 2 (dois) períodos de 5 (cinco) minutos corridos. Persistindo o empate, serão obedecidos os parágrafos anteriores.
	\end{xparagraph}
\end{article}

\begin{article}
	Cada entidade poderá inscrever no máximo 15 (quinze) atletas por partida.
\end{article}

\begin{article}
	Para cada entidade, dentre os jogadores inscritos em súmula apenas 2 (dois) podem ter se formado em 2018 e 2019 e/ou ser alunos de um de seus cursos não estatutários.
\end{article}
