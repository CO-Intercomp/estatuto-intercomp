{\let\clearpage\relax \chapter{Tênis de Mesa}}

\begin{article}
	A modalidade será disputada nas categorias feminina e masculina.
\end{article}

\begin{article}
	Durante os jogos serão obedecidas as regras oficiais, ressalvando os dispostos nos demais artigos deste regulamento.
\end{article}

\begin{article}
	Cada participante poderá inscrever no máximo 4 (quatro) atletas por confronto de Tênis de Mesa, devendo indicar antes do sorteio de emparceiramentos, 2 (dois) atletas para disputar as partidas simples e 2 (dois) atletas para disputar a partida de duplas. Os atletas indicados para disputar as partidas simples serão sorteados no emparceiramento, sendo denominados Atleta A ou Atleta B

	\begin{itemize}[noitemsep]
		\item Uma participante não poderá indicar um único atleta para disputar todas as suas partidas simples.
		\item O mesmo atleta poderá ser indicado por uma participante para disputar partidas simples e partidas de duplas
	\end{itemize}
\end{article}

\noindent
Cada confronto de Tênis de Mesa será disputada como melhor de cinco partidas, sendo declarada vencedora a primeira participante a obter três vitórias. As partidas serão disputadas na seguinte ordem:
\begin{itemize}[noitemsep]
	\item Uma partida entre o Atleta A da participante mandante e o Atleta B da outra participante
	\item Uma partida entre o Atleta B da participante mandante e o Atleta A da outra participante
	\item Uma partida entre as duplas das duas participantes
	\item Uma partida entre o Atleta A da participante mandante e o Atleta A da outra participante
	\item Uma partida entre o Atleta B da participante mandante e o Atleta B da outra participante
\end{itemize}

\noindent
Cada partida de Tênis de Mesa será disputada em 3 (três) sets de 11 (onze) games.
\begin{itemize}[noitemsep]
	\item O set só termina quando um dos participantes obtiver uma vantagem de ao menos 2 (dois) pontos de vantagem sobre o adversário
\end{itemize}

\noindent
Antes de cada partida é sorteado um jogador para iniciar a partida sacando. Em partidas simples o jogador sorteado saca uma vez, seu adversário saca duas vezes e a partir daí os jogadores revezam-se sacando duas vezes cada um. Em partidas de duplas, as duplas alternam-se sacando, e os jogadores de cada dupla revezam-se sacando quando for a vez de sua dupla sacar.

\begin{article}
	Cada entidade deverá apresentar para o representante da competição 3 bolinhas oficias lacradas antes do primeiro confronto da competição.
\end{article}

\begin{article}
	O confronto de Tênis de Mesa só se inicia quando as duas entidades em disputa apresentarem ao menos 2 (dois) atletas, ou quando encerrar-se o período de tolerância de 15 (quinze) minutos.

	\begin{xparagraph}
		Transcorrido um período de tolerância, será declarado W.O. da(s) entidade(s) que não possuir(em) ao menos 2 (dois) atletas presentes para disputa do confronto.
	\end{xparagraph}
\end{article}

\begin{article}
	Fica proibido o uso de instrumento sonoro pelas torcidas nas partidas de Tênis de Mesa, sendo considerados distúrbios de torcida.

	\begin{xparagraph}
		Caso ocorram distúrbios de torcida, a arbitragem ou o representante da competição poderá solicitar a retirada das torcidas do local da competição.
	\end{xparagraph}
\end{article}

\begin{article}
	Não será permitido o uso de madeira rústica ou cortiça como superfície de batida. As raquetes caneteiras que tenham madeira em apenas um dos lados são legais, porém se um jogador golpear a bola com este lado da raquete ele perderá o ponto.
\end{article}
