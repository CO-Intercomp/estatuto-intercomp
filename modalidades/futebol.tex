{\let\clearpage\relax \chapter{Futebol}}

\begin{article}
	A modalidade será disputada na categoria masculina.
\end{article}

\begin{article}
	Durante os jogos serão obedecidas as regras oficiais adotadas pela Federação Paulista de Futebol (FPF) vigentes no início da competição, ressalvando os dispostos nos demais artigos deste regulamento.
\end{article}

\begin{article}
	Os jogos serão compostos de 2 (dois) períodos de 35 (trinta e cinco) minutos cada, com 10 (dez) minutos de descanso entre eles.

	\begin{xparagraph}
		Em caso de empate serão cobrados 5 (cinco) pênaltis alternadamente, por 5 (cinco) jogadores diferentes que terminaram a partida. Persistindo o empate, os pênaltis continuarão a ser cobrados alternadamente sem que haja repetição do atleta cobrador. Após todos os jogadores em campo cobrarem a penalidade máxima, a lista volta ao primeiro jogador.
	\end{xparagraph}

	\begin{xparagraph}
		Em caso de empate no confronto final, será disputada uma prorrogação de 2 (dois) tempos de 10 (dez) minutos. Persistindo o empate, a decisão seguirá o parágrafo anterior.
	\end{xparagraph}
\end{article}

\begin{article}
	Cada entidade poderá inscrever na partida de futebol no máximo 22 (vinte e dois) atletas por partida.
\end{article}

\begin{article}
	Serão permitidas tanto para o tempo regulamentar quanto para a prorrogação 7 (sete) substituições.
\end{article}

\begin{article}
	Para cada entidade, dentre os jogadores inscritos em súmula apenas 5 (cinco) podem estar formados e/ou ser alunos de um de seus cursos não estatutários.
\end{article}
