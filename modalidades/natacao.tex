{\let\clearpage\relax \chapter{Natação}}

\begin{article}
	A modalidade será disputada nas categorias feminina e masculina.
\end{article}

\begin{article}
	Durante os jogos serão obedecidas as regras oficiais, ressalvando os dispostos nos demais artigos deste regulamento.

	Fica definido para a natação o seguinte \textit{Conjunto de Provas}:
	\begin{itemize}[noitemsep]
		\item 50m livres
		\item 50m costas
		\item 50m borboleta
		\item 50m peito
		\item 100m medley
		\item Revezamento 4 x 50m livres
		\item Revezamento 4 x 50m medley
	\end{itemize}

	\begin{xparagraph}
		Ficam definidas como \textit{Provas Individuais} as seguintes provas: 50m livres, 50m costas, 50m borboleta, 50m peito, 100m medley.
	\end{xparagraph}

	\begin{xparagraph}
		Ficam definidas como \textit{Provas por Equipe} as seguintes provas: Revezamento 4 x 50m livres, Revezamento 4 x 50m medley
	\end{xparagraph}
\end{article}

\noindent
Ficam definidas para a natação as seguintes \textit{Regras de Inscrição}:
\begin{itemize}[noitemsep]
	\item Cada entidade poderá inscrever até 2 (dois) atletas em cada \textit{Prova Individual}
	\item Cada entidade poderá inscrever até 1 (uma) equipe em cada \textit{Prova por Equipe}
	\item Cada atleta poderá competir em até 3 (três) \textit{Provas Individuais}
	\item Cada atleta poderá competir em até 2 (duas) \textit{Provas por Equipes}
\end{itemize}

\noindent
Fica definido para a natação o seguinte \textit{Sistema de Classificação}:
\begin{itemize}[noitemsep]
	\item Para o cálculo da pontuação de cada participante, serão considerados os 10 (dez) atletas melhor classificados em cada \textit{Prova Individual}, atribuindo-se a eles a respectiva pontuação: 10, 8, 6, 5, 4, 3, 2, 2, 1, 1 pontos.
	\item Caso uma participante possua mais de um atleta dentre os 10 (dez) melhores classificados de uma \textit{Prova Individual}, a pontuação obtida por cada um dos atletas será somada
	\item Para o cálculo da pontuação de cada participante, serão consideradas as 10 (dez) equipes melhor	classificadas em cada \textit{Prova por Equipes}, atribuindo-se a elas a respectiva pontuação: 20, 16, 12, 10, 8, 6, 5, 4, 3, 2 pontos.
\end{itemize}

\noindent
Só será considerada válida a participação de um atleta em uma \textit{Prova Individual} ou \textit{Prova por Equipes} caso o mesmo complete a distância a ser percorrida em até 1 minuto e 30 segundos (para distâncias de 50 metros) ou 3 minutos (para distâncias de 100 metros), sem infringir qualquer regra daquela modalidade e sem tocar com os pés o fundo da piscina.

\noindent
Só será considerada válida a participação de uma participante caso ela possua ao menos um atleta com participação válida em ao menos uma das modalidades
\begin{itemize}[noitemsep]
	\item As provas seguirão uma sequência decidida pela Comissão Técnica.
\end{itemize}

\begin{article}
	A piscina será liberada para o aquecimento 1 (uma) hora antes do horário estabelecido de início das provas.
\end{article}
