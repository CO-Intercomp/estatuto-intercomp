{\let\clearpage\relax \chapter{Xadrez}}

\begin{article}
	A modalidade será disputada na categoria absoluto.
\end{article}

\begin{article}
	Durante os jogos serão obedecidas às regras oficiais adotadas pela FIDE (World Chess Federation), ressalvando os dispostos nos demais artigos deste regulamento.
\end{article}

\begin{article}
	Cada entidade participante deverá inscrever no máximo 2 (dois)atletas por confronto de xadrez.
	
	\begin{xparagraph}
	    Para cada entidade, dentre os jogadores apenas 1 (um) pode ter se formado em 2018 ou 2019 e/ou ser aluno de um de seus cursos não estatutários.
	\end{xparagraph}
\end{article}

\begin{article}
	A modalidade se dará em formato suíço em melhor de 1 (uma) partida com 6 (seis) rodadas.

	\begin{xparagraph}
		Cada partida de xadrez terá duração máxima de 10 (dez) minutos por jogador
	\end{xparagraph}

	\begin{xparagraph}
		Ao final das rodadas, soma-se as pontuações dos jogadores da equipe, e com isso tem-se a pontuação da equipe. A colocação final se dará pela pontuação da equipe.
	\end{xparagraph}

	\begin{xparagraph}
		Caso ocorra de atletas da mesma delegação se confrontarem, haverá a normalização dos pontos após o final de todas as rodadas.
	\end{xparagraph}

\end{article}

\textbf{Pontuação:}
\begin{table}[h]
\centering
\begin{tabular}{|c|c|c|c|c|c|c|c|}
\hline
lugar & pontuação & lugar & pontuação & lugar & pontuação & lugar & pontuação \\ \hline
1     & 18        & 5     & 12         & 9     & 8         & 13    & 4         \\ \hline
2     & 15        & 6     & 11         & 10    & 7         & 14    & 3         \\ \hline
3     & 14        & 7     & 10         & 11    & 6         & 15    & 2         \\ \hline
4     & 13        & 8     & 9         & 12    & 5         & 16    & 1         \\ \hline
\end{tabular}
\caption{tabela para pontuação do Xadrez}
\end{table}

\begin{article}
	Caso algum jogador não tenha comparecido a o início do confronto de xadrez, será acionado o relógio daquele jogador após o tempo limite de tempo de tolerância no horário marcado para início da partida.
\end{article}


\begin{article}
	É de responsabilidade de cada equipe trazer, a cada confronto, 2 (dois) jogos de peças, 2 (dois) tabuleiros e 2 (dois) relógios de xadrez, em perfeito estado de funcionamento, sendo esse último podendo ser substituido por aplicativos equivalentes de celular.

	\begin{xparagraph}
		Caso, no momento da realização do confronto, não houver material suficiente em condições de uso para o seu início, a entidade que tiver deixado de cumprir a orientação do caput deste artigo será punida com W.O. no confronto.
	\end{xparagraph}
\end{article}

\begin{article}
	Fica proibido o uso de instrumento sonoro no local de disputa do xadrez, sendo considerados distúrbios de torcida.

	\begin{xparagraph}
		Caso ocorram distúrbios de torcida, a arbitragem ou o representante da competição poderá solicitar a retirada das torcidas do local da competição.
	\end{xparagraph}
\end{article}
