{\let\clearpage\relax \chapter{Handebol}}

\begin{article}
	A modalidade será disputada nas categorias feminina e masculina.
\end{article}

\begin{article}
	Durante os jogos serão obedecidas as regras oficiais adotadas pela Confederação Brasileira de Handebol (CBHb), ressalvando os dispostos nos demais artigos deste regulamento.
\end{article}

\begin{article}
	A categoria masculina será realizada em 2 (dois) períodos de 25 (vinte e cinco) minutos, sendo os 2 (dois) últimos minutos cronometrados, com 10 (dez) minutos de intervalo; a categoria feminina será realizada em 2 (dois) períodos de 25 (vinte) minutos, sendo os 2 (dois) últimos minutos cronometrados, com 10 (dez) minutos de intervalo entre eles.

	\begin{xparagraph}
		Em caso de empate, deverá ser realizada cobrança de série de 5 (cinco) tiros de 7 (sete) metros alternados. Persistindo o empate, serão cobrados tiros de 7 (sete) metros alternados até que haja um vencedor, podendo ser repetido o batedor.
	\end{xparagraph}

	\begin{xparagraph}
		Poderão cobrar os tiros de 7 (sete) metros quaisquer jogadores que estejam inscritos na súmula do jogo e não tenham sido expulsos, desqualificados ou que estejam cumprindo exclusão de jogo por 2 (dois) minutos.
	\end{xparagraph}

	\begin{xparagraph}
		Em caso de empate no confronto final, haverá uma prorrogação de 2 (dois) períodos de 5 (cinco) minutos cada. Persistindo o empate, será obedecido o critério adotado no parágrafo acima.
	\end{xparagraph}
\end{article}

\begin{article}
	Cada entidade poderá inscrever no máximo 16 (dezesseis) atletas por partida.
\end{article}

\begin{article}
	Para cada entidade, dentre os jogadores inscritos em súmula apenas 3 (três) podem ter se formado em 2018 e 2019 e/ou ser alunos de um de seus cursos não estatutários.
\end{article}
